\documentclass[12pt, class=report, crop=false]{standalone}
\usepackage{msc_thesis}

% !TeX spellcheck = en-GB
% !TEX bib = reference.bib
% chktex-file 21 # This command might not be intended.

\begin{document}

\chapter{Results}%
\label{chap:results}

Introduction stuff

There are multiple ways of accelerating particles via laser-plasma interactions.
One can classyfy these methods according to the target type, that is if the
target is solid or gaseous (under-dense). From the point of view of the physics, in the
case of a solid, the incoming electromagnetic wave does not propagate inside
the material, while in the case of a under-dense plasma, the laser propagates
through the media. From the point of view of the simulation, the difference is
given by the density of the simulated species. Thus the laser wakefied mechanism
is present only in the case where the laser propagates through the plasma.

\section{Solid target}

We begin with some simulations concernig solid targets following
\textcite{budriga_modelingultrahigh_2017}. For all the simulations we will consider
a plastic cone target interacting with an incoming laser pulse as shown in the
initial snapshot in \cref{fig:initial-plasma-laser}.

\begin{figure}[h]
    \centering
    \includegraphics{}
    \caption{Initial configuration of the system}
    \label{fig:initial-plasma-laser}
\end{figure}

The laser is modeled as a gaussian pulse with

\section{Gaseous target}

\end{document}
