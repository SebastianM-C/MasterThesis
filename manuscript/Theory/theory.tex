\documentclass[../thesis.tex]{subfiles}

%!TeX spellcheck = en-GB

\begin{document}

\chapter{Theory}
\label{chap:Theory}

\section{Classical Electrodynamics}

{\color{red} Introduction stuff, cite~\textcite{eisenberg_nuclear_1978}}.

We will begin with Maxwell's equations
\begin{subequations}
  \label{eq:maxwell}
  \begin{align}
    \div{\vb{E}} & = \frac{\rho}{\varepsilon_0} \label{eq:coulomb-law} \\
    \div{\vb{B}} & = 0 \label{eq:gauss-law} \\
    \curl{\vb{E}} & = - \pdv{\vb{B}}{t} \label{eq:faraday-law} \\
    \curl{\vb{B}} & = \mu_0 \vb{j} + \frac{1}{c^2} \pdv{\vb{E}}{t} \,, \label{eq:ampere-law}
  \end{align}
\end{subequations}

which relate the electromagnetic field to sources, which must satisfy an additional
equation to ensure charge conservation
\[
  \div{\vb{j}(\vb{r}, t)} + \pdv{\rho(\vb{r}, t)}{t} = 0 \,.
\]

As we can see above, equations~\eqref{eq:faraday-law} and~\eqref{eq:gauss-law}
do not involve sources and thus they state the dynamical properties of the fields.
Since equations~\eqref{eq:coulomb-law} and~\eqref{eq:ampere-law} describe how
the sources influence the fields, we need an additional equation to describe how
the fields affect the sources
\[
  \vb{F} = \int \dd{\vb{r}'} \rho(\vb{r}', t) \vb{E}(\vb{r}', t) +
           \frac{1}{c} \int \dd{\vb{r}'} \vb{j}(\vb{r}', t) \cross \vb{B}(\vb{r}', t)\,.
\]

Maxwell's equations~\eqref{eq:maxwell} relate six field quantities (\(\vb{E}\) and \(\vb{B}\))
to four source quantities (\(\rho\) and \(\vb{j}\)). This implies that there are some
restrictions on the six quantities. This suggets that we can find a less redundant
way to express the fields, and indeed the four quantities given by the
vector potential \(\vb{A}\) and scalar potential \(\rho\) provide this representation.
Equation~\eqref{eq:gauss-law} implies the existence of a vector potential
\begin{equation}
  \label{eq:vector-potential}
  \vb{B}(\vb{r}, t) = \curl{\vb{A}(\vb{r}, t)}\,.
\end{equation}

Substituting~\eqref{eq:vector-potential} in~\eqref{eq:faraday-law} we obtain
\begin{equation}
  \label{eq:faraday-vector-potential}
  \curl(\vb{E} + \pdv{\vb{A}}{t}) = 0
\end{equation}

and thus the quantity in the paranthesis can always be expressed as the
gradient of a scalar field, namely the scalar potential
\[
\grad{\phi}(\vb{r}, t) = -\vb{E}(\vb{r}, t) - \pdv{\vb{A}}{t}\,.
\]

With these considerations~\eqref{eq:coulomb-law} becomes
\[
  \div(\grad{\phi} + \pdv{\vb{A}}{t}) = - \frac{\rho}{\varepsilon_0}
\]
or
\begin{equation}
  \label{eq:coulomb-scalar-potential}
  \laplacian{\phi} + \pdv{t}\div{\vb{A}} = - \frac{\rho}{\varepsilon_0}
\end{equation}

and~\eqref{eq:ampere-law}

\begin{equation}
  \label{eq:ampere-potentials-pre}
  \curl(\curl{\vb{A}}) = \mu_0 \vb{j}
    - \frac{1}{c^2} \pdv{t} \left( \grad{\phi} + \pdv{\vb{A}}{t} \right)\,.
\end{equation}

Using the following vector identity
\[
  \curl(\curl{\vb{A}}) = \grad(\div{\vb{A}}) - \laplacian{\vb{A}}
\]

eq.~\eqref{eq:ampere-potentials-pre} becomes
\[
  \grad(\div{\vb{A}}) - \laplacian{\vb{A}} = \mu_0 \vb{j}
    - \frac{1}{c^2} \left( \grad{\pdv{\phi}{t}} + \pdv[2]{\vb{A}}{t} \right)
\]
or
\begin{equation}
  \label{eq:ampere-potentials}
  \laplacian{\vb{A}} - \frac{1}{c^2}\pdv[2]{\vb{A}}{t} =
    -\mu_0 \vb{j} + \grad(\div{\vb{A}} + \frac{1}{c^2} \pdv{\phi}{t})\,.
\end{equation}

Equations~\eqref{eq:coulomb-scalar-potential} and~\eqref{eq:ampere-potentials}
were obtained by substituting the potentials obtained from the source-less
equations,~\eqref{eq:gauss-law} and~\eqref{eq:faraday-law}, into the ones
with sources,~\eqref{eq:coulomb-law} and~\eqref{eq:ampere-law}. They are thus
fully equivalent with Maxwell's equations~\eqref{eq:maxwell} and, as we can observe,
relate the four quantities given by the potentials to the four quantities for the
sources. They also preserve the invariance under Lorentz transformations, with
the scalar potential \(\phi\) as the time-like component.

Equations~\eqref{eq:coulomb-scalar-potential} and~\eqref{eq:ampere-potentials}
can be simplified by decoupling the potentials. This is possible due to the fact
that potentials are not unique. To illustrate this point consider
\[
  \vb{A}'(\vb{r},t) = \vb{A}(\vb{r},t) + \grad{\Lambda}(\vb{r},t)\,.
\]

This vector potential gives rise to a magnetic field
\[
  \curl{\vb{A}'} = \curl{\vb{A}} + \curl(\grad{\Lambda}) = \curl{\vb{A}} = \vb{B}
\]
equal with the original one since \( \curl(\grad{\varphi}) = 0 \).

Similarly, for a scalar potential
\[
  \phi'(\vb{r},t) = \phi(\vb{r},t) - \pdv{\Lambda(\vb{r},t)}{t}
\]
and the corresponding electric field will be
\[
  - \grad{\phi'} - \pdv{\vb{A'}}{t} =
  - \grad{\phi} + \grad{\pdv{\Lambda}{t}} - \pdv{\vb{A}}{t} - \pdv{t} \grad{\Lambda}
  = - \grad{\phi} - \pdv{\vb{A}}{t}
  = \vb{E}\,,
\]
since the spatial and temporal derivatives commute. These kinds of transformations
are called gauge transformations.

\subsection{Gauge transformations}

The freedom of choosing the gauge leads to the following condition satisfied by
the scalar and vector potentials
\[
  \div{\vb{A}} + \frac{1}{c^2}\pdv{\phi}{t} = 0\,,
\]
called the Lorenz condition.

Indeed, if we consider a set of potentials \(\vb{A}\) and \(\phi\) that
don't satisfy the condition
\[
  \div{\vb{A}} + \frac{1}{c^2}\pdv{\phi}{t} \ne 0 = f(\vb{r},t)\,,
\]
then we can always carry out a gauge transformation to a new set of potentials
\(\vb{A}'\) and \(\phi'\) that satisfy the Lorenz condition, such that
\begin{align*}
  \div{\vb{A}} + \frac{1}{c^2}\pdv{\phi}{t} &= \div(\vb{A}' - \grad{\Lambda})
    + \frac{1}{c^2}\pdv{t} \left( \phi' + \pdv{\Lambda}{t} \right) \\
    &= \div{\vb{A}'} - \laplacian{\Lambda} + \frac{1}{c^2}\pdv{\phi'}{t}
    + \frac{1}{c^2}\pdv[2]{\Lambda}{t}
    = f(\vb{r},t)
\end{align*}

or
\[
  \div{\vb{A}} + \frac{1}{c^2}\pdv{\phi}{t} =
  \dalambert \Lambda \equiv \frac{1}{c^2}\pdv[2]{\Lambda}{t} - \laplacian{\Lambda} = f(\vb{r},t)\,,
\]

where the d'Alambertian operator is defined as
\[
  \dalambert \equiv \frac{1}{c^2}\pdv[2]{t} - \laplacian
\]
when choosing the Minkowski metric \( (+,-,-,-) \) and
\[
  \div{\vb{A'}} + \frac{1}{c^2}\pdv{\phi'}{t} = 0\,,
\]
since they satisfy the Lorenz condition.
The transformation we need is thus defined by the solution of \(\dalambert \Lambda = f\).

Imposing the Lorenz condition on equations \eqref{eq:coulomb-scalar-potential}
and \eqref{eq:faraday-vector-potential} decouples the potentials
\begin{align*}
  \laplacian{\phi} - \pdv{t} \frac{1}{c^2} \pdv{\phi}{t} = -\frac{\rho}{\varepsilon_0} \\
  \laplacian{\vb{A}} - \frac{1}{c^2}\pdv[2]{\vb{A}}{t} = -\mu_0 \vb{j}
\end{align*}
yielding the simplified form of Maxwell's equations
\begin{align*}
  \dalambert \phi &= \frac{\rho}{\varepsilon_0} \\
  \dalambert \vb{A} &= \mu_0 \vb{j}\,.
\end{align*}

This form of Maxwell's equations preserves Lorentz invariance, as the Lorenz
gauge condition can be expressed in a covariant way as the contraction of the
four-vector \(A \equiv (\frac{\phi}{c}, \vb{A})\) with the four-gradient
\((\frac{1}{c}\pdv{t}, -\grad)\).

While the Lorenz condition doesn't fix the gauge, but only restricts us to
transformations with \(\dalambert \Lambda = 0\), we can impose further conditions
in order to fix the gauge, but in general those will not be covariant.
One such condition is given by the Coulomb gauge
\begin{equation}
  \label{eq:coulomb-gauge}
  \div{\vb{A}} = 0.
\end{equation}

In this gauge eq.~\eqref{eq:coulomb-scalar-potential} becomes a Poisson equation
for the scalar potential
\begin{equation}
  \label{eq:coulomb-poisson}
  \laplacian{\phi} = -\frac{\rho}{\varepsilon_0}
\end{equation}
with the solution given by
\begin{equation}
  \label{eq:scalar-potential-solution}
  \phi(\vb{r},t) = \int \frac{\rho(\vb{r}',t)}{|\vb{r}-\vb{r}'|} \dd{\vb{r}}'\,,
\end{equation}
explaining the name of the condition~\eqref{eq:coulomb-gauge}.

An apparent violation of special relativity shows up in the above result which
states that the scalar potential (at time \(t\)) is given by the instantaneous Coulomb
interactions between charges (also at time \(t\)). The contradiction is only
apparent and stems from the act that the Coulomb gauge is not Lorentz invariant.

In order to resolve the contradiction we first note that we can only observe
the electric field
\[
  \vb{E}(\vb{r}, t) = -\grad{\phi}(\vb{r},t) -\pdv{\vb{A}(\vb{r},t)}{t}\,.
\]

In the Coulomb gauge, the vector potential is given by
\begin{equation}
  \label{eq:vector-potential-coulomb-gauge}
  \dalambert \vb{A} = \mu_0 \vb{j} - \frac{1}{c^2} \grad{\pdv{\phi}{t}}\,,
\end{equation}
with \(\phi\) being the scalar potential from eq.~\eqref{eq:scalar-potential-solution}.
Considering the continuity equation and the form of the scalar potential in
eq.~\eqref{eq:scalar-potential-solution}, the second term in
eq.~\eqref{eq:vector-potential-coulomb-gauge} becomes
\begin{equation}
  \label{eq:scalar-potential-continuity-equation}
  \grad{\pdv{\phi}{t}} = \grad \int \frac{\pdv{\rho}{t}}{|\vb{r}-\vb{r}'|} \dd{\vb{r}}'
    = -\grad \int \frac{\boldsymbol{\nabla}' \vdot \vb{j}(\vb{r}',t)}{|\vb{r}-\vb{r}'|} \dd{\vb{r}}'\,,
\end{equation}
where \(\boldsymbol{\nabla}'\) denotes the derivatives with respect to \(\vb{r}'\).
Using the Helmholz decomposition we can write the current density as the sum
of a divergence-free (transversal) component and a curl-free (longitudinal) one:
\[
  \vb{j} = \vb{j}^\perp + \vb{j}^\parallel\,,
\]
where
\begin{align*}
  \div{\vb{j}^\perp} &= 0 \\
  \curl{\vb{j}^\parallel} &= 0
\end{align*}
and
% check (and understand) the equations bellow; the current form is probably wrong
\begin{align*}
  \vb{j}^\perp &= \curl \int \frac{\boldsymbol{\nabla}' \cp \vb{j}(\vb{r}',t)}{|\vb{r}-\vb{r}'|} \dd{\vb{r}}' \\
  \vb{j}^\parallel &= \grad \int \frac{\boldsymbol{\nabla}' \vdot \vb{j}(\vb{r}',t)}{|\vb{r}-\vb{r}'|} \dd{\vb{r}}'
\end{align*}

\section{Electron in a Plane Wave}

In this section we will consider the classical dynamics of an electron in a
laser pulse following the discusion in~\textcite{karsch_applications_2018}.
The starting point is the equation of motion for the electron
\begin{equation}
  \label{eq:lorentz-eom}
  \dv{\vb{p}}{t} = -e \left[ \vb{E}(\vb{r}, t) + \vb{v} \cp \vb{B}(\vb{r}, t) \right]\,.
\end{equation}

\subsection{Non-relativistic treatment}

\subsection{Relativistic treatment}

\section{Particle in Cell Method}

\subsection{EPOCH}

\end{document}
