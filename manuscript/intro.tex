\documentclass[12pt, class=report, crop=false]{standalone}
\usepackage{msc_thesis}

% !TeX spellcheck = en-GB
% !TEX bib = reference.bib

\begin{document}

% \addcontentsline{toc}{chapter}{Introduction}
\chapter{Introduction}%
\label{chap:intro}

The advent of high power laser infrastructures and the surge of available
simulation solutions for laser plasma interactions have greatly catalised
the research into physics with high power lasers. Motivates by the
Extreme Light Infrastructure-Nuclear Physics (ELI-NP) experimental facility,
which is currently finalised within ``Horia Hulubei'' National Institute of Physics and Nuclear Engineering, we investigate in this thesis the interaction of high power
laser pulses with solid and gaseous targets.
ELI-NP is one of the three pillars of the ELI Project, which is dedicated to
building a distributed, pan-European laboratory, that hosts beyond state-of-the-art
ultra-short and ultra-intense laser systems. The ELI project is on the roadmap
of European Strategic Forum for Research Infrastructures and is envisaged to produce
outstanding experimental results, such as the 10PW world record reached in March 2019
at ELI-NP~\autocite{eli-npteam_pressrelease_2019}.

The other two ELI centres are ELI-Beamlines, located in the Czech Republic, outside of Prague,
and Hungary, in Szeged.
The interaction of high power lasers with gaseous and solid targets is
of maximal experimental interest at ELI-NP, as can be seen from the
technical design reports~\autocite{gales_introduction_2016}.
The Whitebook of ELI~\autocite{mourou_eliextreme_2011} devotes a few sections
to computing at large, with special emphasis on Particle in Cell codes,
Osiris~\autocite{fonseca_osiristhreedimensional_2002} in particular,
in the context of Neutron sources~\autocite[79]{mourou_eliextreme_2011} and Proton-ion beamlines~\autocite[323]{mourou_eliextreme_2011} amongst others.

Similarly, at the BELLA Petawatt Laser facility at the Lawrence Berkeley National Laboratory
there is a strong
modelling group which develops computational tools for gas dynamics, plasma sources,
laser-plasma acceleration, electromagnetic radiation, beam transport and
laser driven ion acceleration. Among the computing solutions developed at LBNL
we mention here the Warp framework~\autocite{grote_warpcode_2005}, used for
accelerator design and plasma modelling, the Adaptive Mesh Refinement (AMR) library
AMReX~\autocite{zhang_amrexframework_2019} that implements the AMR methodology
and the PICSAR library~\autocite{vincenti_efficientportable_2017} which has
highly optimised implementations of the elementary Particle in Cell operations.
These led to the creation of the Warp-X framework, which combines the previously
mentioned libraries in order to provide an efficient and scalable solution for
simulations on present exa-scale supercomputers~\autocite{vay_warpxnew_2018}.

Our results are obtained using a flavour of PIC algorithms which is known
to describe the physics of PW laser interaction with matter, namely
EPOCH~\autocite{arber_contemporaryparticleincell_2015}. We have surveyed around
a dozen flavours of PIC codes and chose EPOCH due to having advanced physics
features such as support for collisions, multiple ionization modes and QED
effects. Moreover, its numerical noise properties~\autocite{arber_contemporaryparticleincell_2015}
are well documented and the code is open source for academic usage.

EPOCH is a fully relativistic PIC code that can work in the 1, 2 and 3D regimes,
with the field solver using Finite Differences Time Domain (FDTD) algorithms,
supporting multiple order schemes and custom stencils. The particle pusher follows
a modified leapfrog algorithms, having implementations for the well known
Boris push~\autocite{boris_relativisticplasma_1970} and the Higuera-Cary
push~\autocite{higuera_structurepreservingsecondorder_2017}. The code is parallelised
using MPI, features dynamic load balancing and is well integrated with common
visualisation tools.

On the side of physical phenomena, the thesis concerns the so-called laser wakefield
acceleration, which was introduced by~\textcite{tajima_laserelectron_1979}.
When a high intensity laser propagates in plasma, the electric field pushes the
electrons far away from their equilibriunm position (via the ponderomotrive
force). This creates a high intensity field between the electrons
and the nuclei left behind. Any charged particles trapped
in this region will be accelerated.

The rest of the of thesis is structured as follows:
\begin{itemize}
    \item In \cref{chap:physics} some fundamental notions concerning classical
    electrodynamics are presented and the mechanism of the ponderomotive force
    is explained.
    \item In \cref{chap:pic} we survey the core elements of the Particle in Cell
    method, namely the particle pusher and the filed solver and analyse their
    numerical properties.
    \item In \cref{chap:results} the results of this thesis are presented.
    \item In \cref{chap:conclusions} we end with the conclusions.
\end{itemize}

\section*{Acknowledgments}
I would like to thank my advisers, Prof.~dr.~Virgil Băran and Conf.~dr.~Alexandru Nicolin,
for their help. The simulations presented in this thesis were performed on the
computing cluster at Department of Computational Physics and Information
Technologies of the National Institute of Physics and Nuclear Engineering
in close collaboration with dr.~Mihaela Carina Raportaru and PhD student
Teodor Ivanoaica. I would also like to acknowledge fruitful discussions with
the members of the computing group from the Institute of Space Science.

\end{document}
