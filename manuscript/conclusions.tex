\documentclass[12pt, class=report, crop=false]{standalone}
\usepackage{msc_thesis}

% !TeX spellcheck = en-GB
% !TEX bib = reference.bib

\begin{document}

\chapter{Conclusions}%
\label{chap:conclusions}

This thesis addresses the large field of laser plasma interactions in a
computational framework. Following a large theoretical introduction devoted
the fundamentals of classical electrodynamics and Particle in Cell methods, see
\cref{chap:intro,chap:physics,chap:pic}, I present a series of detailed numerical
results on the interaction of high intensity laser pulses with solid and
gaseous targets. The introductory chapters cover in some detail gauge
transformations, the Poynting theorem, electromagnetic waves, the interaction
of a electron with a plane wave, the ponderomotive force and the basics of
laser wakefield acceleration. On the side of the numerical methods, we begin with
an introduction where we present the notions of accuracy and stability and
continue with the main ingredients of the Particle in Cell method, namely the
particle pusher and the field solver. After the theoretical discussion that
highlights the important aspects of stability and conservation properties, we
continue with a survey of concrete PIC implementations and the required HPC
infrastructure for state-of-the art simulations. The fourth chapter of the thesis
is devoted to our results on the interaction of high power laser pulses with
solid and gaseous targets. The numerical simulations reported here were performed
using the EPOCH Particle in Cell code on the computing infrastructure of the
Department of Computational Physics and Information
Technologies of the National Institute of Physics and Nuclear Engineering.

In a nutshell, this thesis is devoted to a class of self-consistent methods of
solving the dynamics of electrically charged particles in strong electromagnetic
fields. The most prominent effect observed in my numerical simulations is the
laser wakefield acceleration visible in the case of gaseous targets subjected
to highly intense laser pulses.

I have found that the interaction of high intensity laser pulse with solid targets,
in our case a plastic cone, impacts substantially less the dynamics of
electrons and protons than in the case of gaseous targets. Note that the
wakefield acceleration does not occur in the case of solid targets since the
laser pulse cannot propagate through the material. This should be contrasted
with gaseous targets where the effect is very visible, to the extent that
it challenges the boundaries of our numerical framework. More specifically,
in the case of gaseous targets the laser wakefield acceleration is so
strong that it leads to an effective depletion of the simulation domain
which is unphysical and is due to the absorptive boundary conditions.

The results reported in this thesis are of interest to the three centers of
the Extreme Light Infrastructure pan-European experimental facility,
currently finalized in Romania, Hungary, and The Czech Republic,
for research activities connected to particle acceleration using lasers.
In the case of accelerating electrons, to a give single examples, a survey of the parameter space will  show the impact on the properties of the electron beams of distinct physical parameters such as: the type and density of gaseous target, the duration and energy of the laser pulse, the acceleration length, the plasma density, etc.
One point of particular interest is the sensitivity of the final results with respect to the spatial and temporal non-uniformities of the laser pulse. The resulting parametric charts will be of immediate interest to the ELI experimental groups.
These numerical investigations will also be focused on the impact of the dimensionality of the codes, the resolution of the numerical grids, the number of super-particles, etc., on the final results.
Two-dimensional codes are particularly appealing when performing parametric scans due to their lower computational load, but they have to be tested for relevance through detailed comparisons will with fully three-dimensional ones.

\end{document}
