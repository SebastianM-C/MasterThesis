\documentclass[12pt, class=report, crop=false]{standalone}
\usepackage{msc_thesis}

% !TeX spellcheck = en-GB
% !TEX bib = reference.bib
% chktex-file 21 # This command might not be intended.

\begin{document}

\chapter{Classical Electrodynamics}%
\label{chap:physics}

In order to study complex phenomena such as laser wakefield acceleration,
we need to have a good understanding of the basic physical phenomena
that govern the dynamics of charged particles in interaction
with electromagnetic fields. In this thesis we will restrict ourselves to
classical electrodynamics, ignoring QED effects that are important
for very high laser intensities \(I \gtrsim \SI{5e22}{\watt/\centi\metre\square}\).
We will mainly follow the ideas presented in~\textcite{jackson_classicalelectrodynamics_1999}
and \textcite[Chapter 2]{eisenberg_nucleartheory_1978}.

We thus begin with Maxwell's equations in free space
\begin{subequations}%
  \label{eq:maxwell}
  \begin{align}
    \div{\vb{E}} & = \frac{\rho}{\varepsilon_0} \label{eq:coulomb-law} \\
    \div{\vb{B}} & = 0 \label{eq:gauss-law} \\
    \curl{\vb{E}} & = - \pdv{\vb{B}}{t} \label{eq:faraday-law} \\
    \curl{\vb{B}} & = \mu_0 \vb{j} + \frac{1}{c^2} \pdv{\vb{E}}{t} \,, \label{eq:ampere-law}
  \end{align}
\end{subequations}

which relate the electromagnetic field to sources. An additional
equation must be satisfied in order to ensure charge conservation
\begin{equation}
  \label{eq:continuity-equation}
  \div{\vb{j}(\vb{r}, t)} + \pdv{\rho(\vb{r}, t)}{t} = 0 \,.
\end{equation}

As we can see above, \cref{eq:faraday-law,eq:gauss-law}
do not involve sources and thus they state the dynamical properties of the fields.
Since \cref{eq:coulomb-law,eq:ampere-law} describe how
the sources influence the fields, we need an additional equation to describe how
the fields affect the sources
\[
  \vb{F} = \int \dd{\vb{r}'} \rho(\vb{r}', t) \vb{E}(\vb{r}', t) +
           \frac{1}{c} \int \dd{\vb{r}'} \vb{j}(\vb{r}', t) \cross \vb{B}(\vb{r}', t)\,.
\]

Maxwell's equations~\eqref{eq:maxwell} relate six field quantities (\(\vb{E}\) and \(\vb{B}\))
to four source quantities (\(\rho\) and \(\vb{j}\)). This implies that there are some
restrictions on the six quantities. This suggests that we can find a less redundant
way to express the fields, and indeed the four quantities given by the
vector potential \(\vb{A}\) and scalar potential \(\rho\) provide this representation.
\Cref{eq:gauss-law} implies the existence of a vector potential
\begin{equation}
  \label{eq:vector-potential}
  \vb{B}(\vb{r}, t) = \curl{\vb{A}(\vb{r}, t)}\,.
\end{equation}

Substituting~\eqref{eq:vector-potential} in~\eqref{eq:faraday-law} we obtain
\begin{equation}
  \label{eq:faraday-vector-potential}
  \curl(\vb{E} + \pdv{\vb{A}}{t}) = 0
\end{equation}

and thus the quantity in the parenthesis can always be expressed as the
gradient of a scalar field, namely the scalar potential
\[
\grad{\phi}(\vb{r}, t) = -\vb{E}(\vb{r}, t) - \pdv{\vb{A}}{t}\,.
\]

With these considerations \cref{eq:coulomb-law} becomes
\[
  \div(\grad{\phi} + \pdv{\vb{A}}{t}) = - \frac{\rho}{\varepsilon_0}
\]
or
\begin{equation}
  \label{eq:coulomb-scalar-potential}
  \laplacian{\phi} + \pdv{t}\div{\vb{A}} = - \frac{\rho}{\varepsilon_0}
\end{equation}

and \cref{eq:ampere-law}

\begin{equation}
  \label{eq:ampere-potentials-pre}
  \curl(\curl{\vb{A}}) = \mu_0 \vb{j}
    - \frac{1}{c^2} \pdv{t} \left( \grad{\phi} + \pdv{\vb{A}}{t} \right)\,.
\end{equation}

Using the following vector identity
\begin{equation}
  \label{eq:curl-of-curl}
  \curl(\curl{\vb{A}}) = \grad(\div{\vb{A}}) - \laplacian{\vb{A}}\,,
\end{equation}

\cref{eq:ampere-potentials-pre} becomes
\[
  \grad(\div{\vb{A}}) - \laplacian{\vb{A}} = \mu_0 \vb{j}
    - \frac{1}{c^2} \left( \grad{\pdv{\phi}{t}} + \pdv[2]{\vb{A}}{t} \right)
\]
or
\begin{equation}
  \label{eq:ampere-potentials}
  \laplacian{\vb{A}} - \frac{1}{c^2}\pdv[2]{\vb{A}}{t} =
    -\mu_0 \vb{j} + \grad(\div{\vb{A}} + \frac{1}{c^2} \pdv{\phi}{t})\,.
\end{equation}

\Cref{eq:coulomb-scalar-potential,eq:ampere-potentials}
were obtained by substituting the potentials obtained from the source-less
equations,~\eqref{eq:gauss-law} and~\eqref{eq:faraday-law}, into the ones
with sources,~\eqref{eq:coulomb-law} and~\eqref{eq:ampere-law}. They are thus
fully equivalent with Maxwell's equations~\eqref{eq:maxwell} and, as we can observe,
relate the four quantities given by the potentials to the four quantities for the
sources. They also preserve the invariance under Lorentz transformations, with
the scalar potential \(\phi\) as the time-like component.

\Cref{eq:coulomb-scalar-potential,eq:ampere-potentials}
can be simplified by decoupling the potentials. This is possible due to the fact
that potentials are not unique. To illustrate this point consider
\[
  \vb{A}'(\vb{r},t) = \vb{A}(\vb{r},t) + \grad{\Lambda}(\vb{r},t)\,.
\]

This vector potential gives rise to a magnetic field
\[
  \vb{B}' = \curl{\vb{A}'} = \curl{\vb{A}} + \curl(\grad{\Lambda})
          = \curl{\vb{A}} = \vb{B}
\]
equal with the original one since \( \curl(\grad{\varphi}) = 0 \).

Similarly, for a scalar potential
\[
  \phi'(\vb{r},t) = \phi(\vb{r},t) - \pdv{\Lambda(\vb{r},t)}{t}
\]
the corresponding electric field will be
\[
  \vb{E}' =
  - \grad{\phi'} - \pdv{\vb{A'}}{t} =
  - \grad{\phi} + \grad{\pdv{\Lambda}{t}} - \pdv{\vb{A}}{t} - \pdv{t} \grad{\Lambda}
  = - \grad{\phi} - \pdv{\vb{A}}{t}
  = \vb{E}\,,
\]
since the spatial and temporal derivatives commute. These kinds of transformations
are called gauge transformations.

\section{Gauge transformations}

The freedom of choosing the gauge leads to the following condition satisfied by
the scalar and vector potentials
\[
  \div{\vb{A}} + \frac{1}{c^2}\pdv{\phi}{t} = 0\,,
\]
called the Lorenz condition.

Indeed, if we consider a set of potentials \(\vb{A}\) and \(\phi\) that
don't satisfy the condition
\[
  \div{\vb{A}} + \frac{1}{c^2}\pdv{\phi}{t} \ne 0 = f(\vb{r},t)\,,
\]
then we can always carry out a gauge transformation to a new set of potentials
\(\vb{A}'\) and \(\phi'\) that satisfy the Lorenz condition, such that
\begin{align*}
  \div{\vb{A}} + \frac{1}{c^2}\pdv{\phi}{t} &= \div(\vb{A}' - \grad{\Lambda})
    + \frac{1}{c^2}\pdv{t} \left( \phi' + \pdv{\Lambda}{t} \right) \\
    &= \div{\vb{A}'} - \laplacian{\Lambda} + \frac{1}{c^2}\pdv{\phi'}{t}
    + \frac{1}{c^2}\pdv[2]{\Lambda}{t}
    = f(\vb{r},t)
\end{align*}

or
\[
  \div{\vb{A}} + \frac{1}{c^2}\pdv{\phi}{t} =
  \dalambert \Lambda \equiv \frac{1}{c^2}\pdv[2]{\Lambda}{t} - \laplacian{\Lambda} = f(\vb{r},t)\,,
\]

where the d'Alambertian operator is defined as
\[
  \dalambert \equiv \frac{1}{c^2}\pdv[2]{t} - \laplacian
\]
when choosing the Minkowski metric \( (+,-,-,-) \) and
\[
  \div{\vb{A'}} + \frac{1}{c^2}\pdv{\phi'}{t} = 0\,,
\]
since they satisfy the Lorenz condition.
The transformation we need is thus defined by the solution of \(\dalambert \Lambda = f\).

Imposing the Lorenz condition on equations~\eqref{eq:coulomb-scalar-potential}
and~\eqref{eq:faraday-vector-potential} decouples the potentials
\begin{align*}
  \laplacian{\phi} - \pdv{t} \frac{1}{c^2} \pdv{\phi}{t} = -\frac{\rho}{\varepsilon_0} \\
  \laplacian{\vb{A}} - \frac{1}{c^2}\pdv[2]{\vb{A}}{t} = -\mu_0 \vb{j}
\end{align*}
yielding the simplified form of Maxwell's equations
\begin{align*}
  \dalambert \phi &= \frac{\rho}{\varepsilon_0} \\
  \dalambert \vb{A} &= \mu_0 \vb{j}\,.
\end{align*}

This form of Maxwell's equations preserves Lorentz invariance, since the Lorenz
gauge condition can be expressed in a covariant way as the contraction of the
four-vector \(A \equiv (\frac{\phi}{c}, \vb{A})\) with the four-gradient
\((\frac{1}{c}\pdv{t}, -\grad)\).

Since the Lorenz condition doesn't fix the gauge, but only restricts us to
transformations with \(\dalambert \Lambda = 0\), we can impose further conditions
in order to fix the gauge, but in general those will not be covariant.
One such condition is given by the Coulomb gauge
\begin{equation}
  \label{eq:coulomb-gauge}
  \div{\vb{A}} = 0.
\end{equation}

In this gauge \cref{eq:coulomb-scalar-potential} becomes a Poisson equation
for the scalar potential
\begin{equation}
  \label{eq:coulomb-poisson}
  \laplacian{\phi} = -\frac{\rho}{\varepsilon_0}
\end{equation}
with the solution given by the instantaneous Coulomb potential of the charge
density in the domain \(\rho(\vb{r},t)\)
\begin{equation}
  \label{eq:scalar-potential-solution}
  \phi(\vb{r},t) = \frac{1}{4\pi \varepsilon_0} \int \frac{\rho(\vb{r}',t)}{|\vb{r}-\vb{r}'|} \dd{\vb{r}}'\,,
\end{equation}
explaining the name of the condition~\eqref{eq:coulomb-gauge}.

An apparent violation of special relativity shows up in the above result which
states that the scalar potential (at time \(t\)) is given by the instantaneous Coulomb
interactions between charges (also at time \(t\)). The contradiction is only
apparent and stems from the act that the Coulomb gauge is not Lorentz invariant.

In order to resolve the contradiction we first note that we can only observe
the electric field
\[
  \vb{E}(\vb{r}, t) = -\grad{\phi}(\vb{r},t) -\pdv{\vb{A}(\vb{r},t)}{t}\,.
\]
Thus, the instantaneous propagation is removed by the time derivative
of the vector potential.

In the Coulomb gauge, the vector potential is given by
\begin{equation}
  \label{eq:vector-potential-coulomb-gauge}
  \dalambert \vb{A} = \mu_0 \vb{j} - \frac{1}{c^2} \grad{\pdv{\phi}{t}}\,.
\end{equation}

Considering the continuity equation~\eqref{eq:continuity-equation} and
the form of the scalar potential in \cref{eq:scalar-potential-solution},
the second term in \cref{eq:vector-potential-coulomb-gauge} becomes
\begin{equation}
  \label{eq:scalar-potential-continuity-equation}
  \grad{\pdv{\phi}{t}} = \grad \frac{1}{4\pi \varepsilon_0}
    \int \frac{\pdv{\rho}{t}}{|\vb{r}-\vb{r}'|} \dd{\vb{r}}'
    = - \frac{1}{4\pi \varepsilon_0} \grad
    \int \frac{\boldsymbol{\nabla}' \vdot \vb{j}(\vb{r}',t)}{|\vb{r}-\vb{r}'|} \dd{\vb{r}}'\,,
\end{equation}
where \(\boldsymbol{\nabla}'\) denotes the derivatives with respect to \(\vb{r}'\).
Using the Helmholtz decomposition we can write any sufficiently well behaved
vector (the current density in this particular case) as the sum
of a divergence-free (transversal) component and a curl-free (longitudinal) one:
\[
  \vb{j} = \vb{j}^t + \vb{j}^l \,,
\]
where
\begin{align*}
  \div{\vb{j}^t} &= 0 \\
  \curl{\vb{j}^l} &= 0 \,.
\end{align*}

Using the vector identity~\eqref{eq:curl-of-curl} and
\[
  \laplacian{\frac{1}{|\vb{r}-\vb{r}'|}} = -4\pi \delta(\vb{r}-\vb{r}')
\]
we can write the current density as follows
\begin{align*}
  \laplacian(\vb{j}^t + \vb{j}^l) &= \grad(\div{\vb{j}^l}) - \curl(\curl{\vb{j}^t}) \\
  \int \frac{\laplacian{\vb{j}}}{|\vb{r}-\vb{r}'|} \dd{\vb{r}'} &=
    \int \frac{\grad(\div{\vb{j}^l})}{|\vb{r}-\vb{r}'|} \dd{\vb{r}'}
    - \int \frac{\curl(\curl{\vb{j}^t})}{|\vb{r}-\vb{r}'|} \dd{\vb{r}'} \\
  -4\pi \vb{j} &=
  \underbrace{\grad \int \frac{\div{\vb{j}^l}}{|\vb{r}-\vb{r}'|} \dd{\vb{r}'}}_{-4\pi \vb{j}^l}
    - \underbrace{\curl \curl \int \frac{\vb{j^t}}{|\vb{r}-\vb{r}'|} \dd{\vb{r}'}}_{4\pi \vb{j}^t}
\end{align*}
and thus we obtain the two components as

\begin{align*}
  \vb{j}^t &= \frac{1}{4\pi} \curl \curl \int \frac{\vb{j}(\vb{r}',t)}{|\vb{r}-\vb{r}'|} \dd{\vb{r}}' \\
  \vb{j}^l &= -\frac{1}{4\pi} \grad \int
    \frac{\boldsymbol{\nabla}' \vdot \vb{j}(\vb{r}',t)}{|\vb{r}-\vb{r}'|} \dd{\vb{r}}'\,.
\end{align*}

Comparing with \cref{eq:scalar-potential-continuity-equation} we see that
\[
  \frac{1}{c^2} \grad{\pdv{\phi}{t}} = \frac{\varepsilon_0}{c^2} \vb{j}^l
  = \mu_0 \vb{j}^l
\]
and thus the source term in \cref{eq:vector-potential-coulomb-gauge} can
be expressed as function of the transverse current:
\[
  \dalambert \vb{A} = \mu_0 (\vb{j} - \vb{j}^l) = \mu_0 \vb{j}^t\,
\]
and this also why the Coulomb gauge is also called the transverse gauge.
This gauge is useful when no sources are present. In this case \(\phi=0\),
\(\vb{A}\) satisfies the homogeneous wave equation and the fields can
be expressed only as function of the vector potential
\begin{align*}
  \vb{E} &= -\pdv{\vb{A}}{t} \\
  \vb{B} &= \curl{\vb{A}}\,.
\end{align*}

\section{The Poynting theorem}

In order to complete the description of the interaction between fields
and sources, we will now focus on how the fields affect the particles.
We begin by considering the force acting on a charge \(q\)
\[
  \vb{F} = q\vb{E} + q\vb{v} \cp \vb{B}\,.
\]

The corresponding infinitesimal variation of the force is given by
\[
  \var{\vb{F}} = \rho \vb{E} \var{V} + \vb{j} \cp \vb{B} \var{V}
    = (\rho \vb{E} + \vb{j} \cp \vb{B}) \var{V}
    \equiv f \var{V},
\]
where \(f = \rho \vb{E} + \vb{j} \cp \vb{B}\) is the Lorentz force
density. We can now consider a uniform charge distribution characterized
by \(\rho\). For an infinitesimal volume \(\var{V}\) of this charge distribution,
the rate of change of the work, or the power given by the
fields is given by
\[
  \vb{v} \vdot \vb{F} = \rho \vb{v} \vb{E} + \frac{\vb{j}}{q} \vdot (\vb{j} \cp \vb{B})
    = \rho \vb{v} \vb{E}\,.
\]

As we can see above, the magnetic force doesn't contribute to the work
done by the fields. Thus, the power transferred from the fields to the
charges in a finite domain \(\mathscr{D}\) is
\[
  \int_{\mathscr{D}} \vb{j} \vdot \vb{E} \dd{\vb{r}}\,.
\]

For the energy to be conserved, this power must be balanced by a corresponding
rate of decrease of energy in the electromagnetic field.
Using Ampère law~\eqref{eq:ampere-law}
\[
  \int_{\mathscr{D}} \vb{j} \vdot \vb{E} \dd{\vb{r}} =
  \int_{\mathscr{D}} \vb{E} \vdot \frac{1}{\mu_0} \qty(\curl{\vb{B}} - \frac{1}{c^2} \pdv{\vb{E}}{t}) \dd{\vb{r}} =
  \frac{1}{\mu_0} \int_{\mathscr{D}} \left[\vb{E} \vdot(\curl{\vb{B}}) -
    \frac{1}{c^2} \vb{E} \vdot \pdv{\vb{E}}{t} \right] \dd{\vb{r}}
\]

Using the following vector identity
\[
  \div(\vb{E} \cp \vb{B}) = \vb{B} \vdot (\curl{E}) - \vb{E} \vdot (\curl{\vb{B}})
\]
we can express \(\vb{E} \vdot(\curl{\vb{B}})\) as
\[
  \vb{E} \vdot(\curl{\vb{B}}) = \vb{B} \vdot (\curl{E}) - \div(\vb{E} \cp \vb{B}) =
  - \vb{B} \vdot \pdv{\vb{B}}{t} - \div(\vb{E} \cp \vb{B})\,,
\]
where we used \cref{eq:faraday-law} for the first term.

With these considerations, the power transferred by the fields is given by
\[
  \int_{\mathscr{D}} \vb{j} \vdot \vb{E} \dd{\vb{r}} =
  -\int_{\mathscr{D}} \left[\frac{1}{\mu_0} \div(\vb{E} \cp \vb{B}) + \frac{1}{\mu_0} \vb{B} \vdot \pdv{\vb{B}}{t} +
    \frac{1}{\mu_0 c^2} \vb{E} \vdot \pdv{\vb{E}}{t} \right] \dd{\vb{r}}
\]

Considering that
\[
  \vb{E} \vdot \pdv{\vb{E}}{t} = \frac{1}{2} \pdv{t} \vb{E}^2\,,
\]
we obtain
\[
  \int_{\mathscr{D}} \vb{j} \vdot \vb{E} \dd{\vb{r}} =
  -\int_{\mathscr{D}} \left[ \frac{1}{2} \pdv{t}
    \left( \varepsilon_0 \vb{E}^2 + \frac{1}{\mu_0}\vb{B}^2 \right)
  + \frac{1}{\mu_0}\div(\vb{E} \cp \vb{B}) \right] \dd{\vb{r}}\,.
\]

The total energy density of the electromagnetic field can be denoted with
\[
  w_{em} = \frac{1}{2} \left( \varepsilon_0 \vb{E}^2 + \frac{1}{\mu_0}\vb{B}^2 \right)
\]
and thus we obtain
\[
  -\int_{\mathscr{D}} \vb{j} \vdot \vb{E} \dd{\vb{r}} =
  \int_{\mathscr{D}} \left[ \pdv{w_{em}}{t} +
  \frac{1}{\mu_0}\div(\vb{E} \cp \vb{B}) \right] \dd{\vb{r}}\,.
\]

Since the domain \(\mathscr{D}\) is arbitrary, we can write the above as a
differential continuity equation
\begin{equation}
  \label{eq:poynting-theorem-local}
  \pdv{w_{em}}{t} = - \div{\vb{S}} - \vb{j} \vdot \vb{E}\,,
\end{equation}
where
\[
  \vb{S} = \frac{1}{\mu_0} \vb{E} \cp \vb{B}
\]
is the Poynting vector representing the energy flow.

If we consider the domain \(\mathscr{D}\) such that no particles will leave it
\[
  W_{em} = \int_{\mathscr{D}} w_{em} \dd{\vb{r}}
\]
is the energy of the electromagnetic field and \(W_{mech}\) is the
energy of the particles
\[
  W_{mech} = \int_{\mathscr{D}} w_{mech} \dd{\vb{r}} =
  \int_{\mathscr{D}} \vb{j} \vdot \vb{E} \dd{\vb{r}}\,.
\]

By using Gauss' theorem, the energy flux corresponding to the
Poynting vector becomes
\[
  \int_{\mathscr{D}} \div{\vb{S}} = \oint_\Sigma \vb{n} \vdot \vb{S} \dd{a}\,,
\]
where \(\Sigma\) is the surface enclosing the domain \(\mathscr{D}\).

With the above considerations Poynting's theorem gives the conservation of energy for the whole system
\begin{equation}
  \label{eq:poynting-theorem}
  \dv{W}{t} = \dv{t} \left( W_{em} + W_{mech} \right) =
  -\oint_\Sigma \vb{n} \vdot \vb{S} \dd{a}\,,
\end{equation}

stating that the rate of change of the energy of the system composed of the charged
particles and corresponding fields is given by minus the flux of the
Poynting vector through the surface bounding the domain.

\Cref{eq:poynting-theorem-local} is the local form for the Poynting
theorem.

If we consider the extension of the domain to infinity \(\mathscr{D} \to \mathbb{R}^3\),
\(\Sigma \to \Sigma_\infty\), then there is no energy flow through the
boundary since electromagnetic waves propagate at a constant finite speed \(c\). Then
\[
  \dv{W}{t} = \dv{t} \left( W_{em} + W_{mech} \right) = 0
\]
and the entire energy of the electromagnetic field can be converted into
the mechanical energy of the particles interacting with the field.

If we consider \(\mathscr{D}\) such that it doesn't enclose any sources, than
\[
  \dv{t} W_{em} = -\oint_\Sigma \vb{n} \vdot \vb{S} \dd{a}\,,
\]
which shows that the energy of the electromagnetic field in the domain
\(\mathscr{D}\) can change through the variation of the flux of the
Poynting vector on the boundary of the domain, \(\Sigma\). Thus we can
indeed say that the flux of the Poynting vector is the energy flux.

\section{Electromagnetic waves}

In the absence of sources, Maxwell's equations become
\begin{subequations}
  \begin{align*}
    \div{\vb{B}} &= 0  \qquad \curl{\vb{E}} + \pdv{\vb{B}}{t} = 0 \\
    \div{\vb{D}} &= 0  \qquad \curl{\vb{H}} - \pdv{\vb{D}}{t} = 0\,,
  \end{align*}
\end{subequations}
where \(\vb{D} = \varepsilon \vb{E}\), \(\vb{B} = \mu \vb{H}\) and \(\vb{D}\) is
the displacement field and \(\vb{H}\) is the magnetizing field.
In our case, \(\varepsilon = \varepsilon_0\) and \(\mu = \mu_0\).

If we assume that the time dependence for the solutions is given by
\(\ee^{-\ii \omega t}\), then the above equations become
\begin{subequations}
  \begin{align*}
    \div{\vb{B}} &= 0  \qquad \curl{\vb{E}}\ee^{-\ii \omega t} - \ii \omega \vb{B} \ee^{-\ii \omega t} = 0 \\
    \div{\vb{D}} &= 0  \qquad \curl{\vb{H}}\ee^{-\ii \omega t} + \ii \omega \vb{D} = 0\,.
  \end{align*}
\end{subequations}
More complex time dependencies can be treated with a Fourier decomposition since
if we have a solution, any linear combinations with that solution are also solutions.

If we consider only the curl equations
\[
\begin{aligned}
  \curl{\vb{E}} - \ii \omega \vb{B} = 0 \\
  \curl{\vb{B}} + \ii \omega \mu \epsilon \vb{E} = 0
\end{aligned}
\]
and take the curl
\[
\curl{\curl{\vb{E}}} - \ii \omega \curl{\vb{B}} = \grad{\underbrace{(\div{\vb{E}})}_{0}}
- \laplacian{\vb{E}} + {(\ii \omega)}^2 \mu \varepsilon \vb{E} = 0
\]
we obtain the Helmholtz wave equations
\[
\begin{aligned}
  \left(\laplacian + \omega^2 \mu \varepsilon \right)\vb{E} &= 0 \\
  \left(\laplacian + \omega^2 \mu \varepsilon \right)\vb{B} &= 0
\end{aligned}
\]
or in a more compact notation
\begin{equation*}
\label{eq:helmholtz-wave-eq}
\left(\laplacian + \omega^2 \mu \varepsilon \right)
\begin{pmatrix}
  \vb{E} \\ \vb{B}
\end{pmatrix}
= 0\,.
\end{equation*}

Taking the plane wave
\[
\ee^{\ii k x - \ii \omega t}
\]
as a solution for the Helmholtz wave equation~\eqref{eq:helmholtz-wave-eq}, we can study
what properties arise for wave number \(k\) and the frequency \(\omega\).
Thus we can see that
\[
-k^2 + \omega^2 \mu \varepsilon = 0
\]
or
\[
k = \omega \sqrt{\mu \varepsilon}\,.
\]

The phase velocity of a wave \(v_\phi\) is defined as
\[
v_\phi = \frac{\omega}{k}\,.
\]

In this case, the phase velocity of a plane wave satisfying the
Helmholtz equation is
\[
v_\phi = \frac{1}{\sqrt{\mu\varepsilon}} = \frac{c}{n}\,,
\]
where \(n\) is the refraction index of the media in which the wave propagates and is given by
\[
n = \sqrt{\frac{\mu}{\mu_0}\frac{\varepsilon}{\varepsilon_0}}\,.
\]

A general solution for the wave equation can be constructed by
using the superposition principle
\[
u(x,t) = a \ee^{\ii k x - \ii \omega t} + b \ee^{-\ii k x - \ii \omega t}\,.
\]
This general solution can be seen as a superposition of incoming and outgoing plane waves.

Let us now consider an electromagnetic plane wave, that is a
plane wave that satisfies both the Helmholtz wave equation~\eqref{eq:helmholtz-wave-eq}
and Maxwell's equations~\eqref{eq:maxwell}
\[
\begin{aligned}
  \vb{E}(\vb{x},t) &= \vb*{\mathcal{E}} \ee^{\ii k \vb{n} \vdot \vb{x} - \ii \omega t} \\
  \vb{B}(\vb{x},t) &= \vb*{\mathcal{B}} \ee^{\ii k \vb{n} \vdot \vb{x} - \ii \omega t}\,.
\end{aligned}
\]

First, let us consider the Laplacian in the Helmholtz equation
\[
\pdv[2]{x_i} \mathcal{E}_i \ee^{\ii k n_j x_j - \ii \omega t} =
\pdv{x_i} \ii k n_j \delta_{ij} \mathcal{E}_i \ee^{\ii k n_j x_j - \ii \omega t} =
-k^2 n_i n_i \mathcal{E}_i \ee^{\ii k n_j x_j - \ii \omega t}\,.
\]

With this consideration, Helmholtz equation yields
\[
-k^2 n_i n_i \mathcal{E}_i \ee^{\ii k n_j x_j - \ii \omega t}
+ \mu\varepsilon\omega^2\mathcal{E}_i \ee^{\ii k n_j x_j - \ii \omega t} = 0
\]
or
\[
k^2 \vb{n} \vdot \vb{n} = \mu\varepsilon\omega^2\,.
\]

Considering that \(k=\sqrt{\mu\varepsilon} \omega\), we obtain
that the norm of \(\vb{n}\) must be 1
\[
\vb{n}\vdot\vb{n}=1\,.
\]

Now we continue with Maxwell's equations. We begin with the divergence
equations
\[
\div{\vb{B}} = 0 \implies \partial_i \mathcal{B}_i \ee^{\ii k n_j x_j - \ii \omega t} = 0\,,
\]
where we have used the notation
\[
\partial_i \equiv \pdv{x_i}\,.
\]
Thus
\(
\ii k n_i \mathcal{B}_i = 0
\)
and the magnetic field \(\vb{B}\) and \(\vb{n}\) are perpendicular since
\[
\vb{n} \vdot \vb*{\mathcal{B}} = 0\,.
\]

Similarly, for the electric field
\[
\div{\vb{D}} = 0 \implies \partial_i \varepsilon \mathcal{E}_i \ee^{\ii k n_j x_j - \ii \omega t} = 0
\]
and
\[
\ii k n_i \mathcal{E}_i = 0 \implies \vb{n} \vdot \vb*{\mathcal{E}}=0\,.
\]
Thus, since the electric and magnetic fields are perpendicular to
the propagation direction given by the \(\vb{n}\) versor, the
electromagnetic plane wave is a transverse wave.

If we now consider the curl equation
\[
\curl{\vb{E}} = -\pdv{\vb{B}}{t}\,,
\]
we obtain
\[
\begin{aligned}
  \varepsilon_{ijk} \partial_j E_k &= -(-\ii \omega) \mathcal{B}_i
  \ee^{\ii k n_i x_i- \ii \omega t} \\
  \varepsilon_{ijk} \partial_j \mathcal{E}_k \ee^{\ii k n_i x_i- \ii \omega t} &= \ii \omega \mathcal{B}_i
  \ee^{\ii k n_i x_i- \ii \omega t} \\
  \varepsilon_{ijk} \mathcal{E}_k \ii k n_j &= \ii \omega \mathcal{B}_i\,.
\end{aligned}
\]

Thus \(\vb{n} \cp \vb*{\mathcal{E}} \sqrt{\mu\varepsilon} = \vb*{\mathcal{B}}\) or
\[
\vb{n} \cp \vb*{\mathcal{E}} \frac{n}{c} = \vb*{\mathcal{B}}\,.
\]
In vacuum (\(n=1\)), we observe that the electric and magnetic field differ in magnitude by a factor of \(c\)
\begin{equation}
  \label{eq:e-b-magnitude-comparison}
  \abs{\vb*{\mathcal{E}}} = \abs{c\vb*{\mathcal{B}}}\,.
\end{equation}

If we consider a coordinate system spanned by the versors
\((\vb*{\epsilon}_1,\vb*{\epsilon}_2,\vb{n})\), then the electric
and magnetic field magnitudes can be written as
\[
\vb*{\mathcal{E}} = \vb*{\epsilon}_1 E_0 \qquad
\vb*{\mathcal{B}} = \vb*{\epsilon}_2 \sqrt{\mu\varepsilon} E_0\,.
\]
A plane wave with its electric field always in a direction \(\vb*{\epsilon}_1\) is called \emph{linearly polarized} with the
polarization vector \(\vb*{\epsilon}_1\).

According to the Poynting theorem, the plane wave transports energy. The Poynting vector in this case is given by
\[
\vb{S} = \vb{E} \cp \vb{H} =
\vb{e}_i \varepsilon_{ijk} {\vb*{\epsilon}_1}_j E_0 {\vb*{\epsilon}_2}_k \frac{1}{\mu} \sqrt{\mu\varepsilon} =
\sqrt{\frac{\varepsilon}{\mu}} E_0^2\vb{n}
\]

\section{Electron in a Plane Wave}

In this section we will consider the classical dynamics of an electron in a linearly polarized
laser pulse following the discussion in~\textcite{karsch_applicationshigh_2018}.
We will assume an electron in vacuum and we will consider that the fields vary only along the \(Ox\) direction
and explore the motion in the non-relativistic and relativistic
cases.
Thus the vector potential will be
\begin{equation}
  \label{eq:simple-laser-A}
  \vb{A}(x,t) = \vb{e}_y A_0 \sin(kx - \omega t)
\end{equation}
and the electric and magnetic fields will be given by
\[
\begin{aligned}
  \vb{E}(x,t) = -\pdv{\vb{A}}{t} -\underbrace{\grad{\phi}}_0
  = \vb{e}_y E_0 \cos(kx - \omega t) \\
  \vb{B}(x,t) = \pdv{A_y}{x}\vb{e}_z = \vb{e}_z B_0 \cos(kx - \omega t)\,,
\end{aligned}
\]
where \(E_0=\omega A_0\) and \(B_0=k A_0\).

\subsection{Non-relativistic treatment}

In the non-relativistic case, the Lorentz force that gives the equation of motion for the electron is described by
\begin{equation}
  \label{eq:lorentz-eom}
  \dv{\vb{p}}{t} = -e \left[ \vb{E}(x, t) + \vb{v} \cp \vb{B}(x, t) \right]\,.
\end{equation}

As we noted in \cref{eq:e-b-magnitude-comparison}, the magnetic field amplitude is \(c\) times less than the one
of the electric field and thus the second term in \cref{eq:lorentz-eom} can be ignored, since its magnitude
is proportional to \(v/c\), and in the non-relativistic case
this ratio goes to 0. The equation of motion can be now solved by integration yielding
\[
\vb{p} = -e \int \vb{E}(x,t) \dd t\,.
\]
Considering the electron is at rest at \(t=0\), its velocity
will be given by
\[
\vb{v} = \vb{e}_y \frac{e E_0}{\omega m_e} \sin(kx - \omega t)\,.
\]
We can observe that the maximum velocity for the electron in this case is given by
\[
v_{max} = \frac{e E_0}{\omega m_e}\,. % chktex 35
\]
This gives us an estimate for how far can the non-relativistic approximation can go
since at \(v_{max}=c\) it will no longer be valid  % chktex 35
(actually it will be erroneous well before reaching the breaking point).
Thus, at the breaking point, the relativistic factor \(\beta = v/c\) will be unity and
\[
\beta_{max} \equiv \frac{v_{max}}{c} = \frac{e E_0}{\omega m_e c} = \frac{e A_0}{m_e c} \equiv a_0\,,  % chktex 35
\]
where \(a_0\) is called the normalized vector potential and
can be used to measure how ``relativistic'' is the electron motion in the corresponding electric field.
We can thus use it to estimate the laser intensities up to which the non-relativistic description is valid.
Considering \(a_0=1\) we express the electric field magnitude as
\[
E_0 = a_0 \frac{\omega m_e c}{e} = \frac{a_0}{\lambda} 2\pi \frac{m_e c^2}{e}\,.
\]

Since laser wavelengths are usually expressed in \si{\micro\metre}, it can be useful to write the above as
\[
E_0 = \frac{a_0}{\lambda} \SI{3.21e12}{\volt\per\metre\micro\metre}\,.
\]

Similarly, the magnetic field will be given by
\[
B_0 = \frac{E_0}{c} = \frac{a_0}{\lambda} \SI{1.07e18}{\tesla\micro\metre}\,.
\]

Thus the limiting laser intensity for the non-relativistic description will be around
\[
I = \frac{\varepsilon_0 c}{2} E_0^2 =
\frac{a_0^2}{\lambda^2} \SI{1.37e18}{\watt\per\centi\metre\squared\micro\metre\squared}\,,
\]
where we have used \cref{eq:plane-wave-intensity}.
Thus for laser intensities above \SI{e18}{\watt\per\centi\metre\squared},
the motion must be treated relativistically.

\subsection{Relativistic treatment}

We will now continue with the relativistic case, where the equation
of motion is changed by the fact that the momentum of the electron is given by
\[
\vb{p} = \gamma m_e \vb{v}\,,
\]
where
\[
\gamma = \frac{1}{\sqrt{1-\frac{v^2}{c^2}}} = \frac{1}{1-\beta^2}\,.
\]

Thus the equation of motion will be
\begin{equation}
  \label{eq:relativistic-lorentz}
  \dv{\vb{p}}{t} = \dv{t}(\gamma m_e \vb{v}) = -e (\vb{E} + \vb{v}\cp\vb{B})\,.
\end{equation}

The \(\gamma\) factor can also be written as a function of the momenta
\[
\begin{aligned}
  \gamma &= \frac{1}{\sqrt{1-\frac{v^2}{c^2}}} \implies
  \gamma^2\left(1-\frac{v^2}{c^2}\right)=1\\ \text{ or }
  \gamma^2 &= 1+\gamma^2\frac{v^2}{c^2}\\
  \gamma &= \sqrt{1+\gamma^2\frac{v^2}{c^2}} \implies
  \gamma = \sqrt{1 + {\left(\frac{\vb{p}}{m_e c}\right)}^2}\,.
\end{aligned}
\]
The time derivative of the \(\gamma\) factor is thus given by
\[
\dv{\gamma}{t} = \dv{t}\sqrt{1+{\left(\frac{\vb{p}}{m_e c}\right)}^2} =
\frac{1}{\gamma} {\dv{t}} {\left(\frac{\vb{p}}{m_e c}\right)}^2 =
\frac{1}{2\gamma}\frac{1}{{(m_e c)}^2} \dv{\vb{p}^2}{t}\,,
\]
where the time derivative of the square of the momenta can also be expressed as
\[
\frac{1}{2}\dv{\vb{p}^2}{t} = \vb{p}\vdot\dv{\vb{p}}{t} =
-e\vb{p}\vdot\vb{E}-e\underbrace{\vb{p}\vdot(\vb{v}\cp\vb{B})}_0 =
-e\vb{p}\vdot\vb{E}\,.
\]

With this consideration, the time derivative for \(\gamma\) yields
\begin{equation}
  \label{eq:gamma-time-derivative}
  \dv{\gamma}{t} = \frac{1}{\gamma} \frac{1}{{(m_e c)}^2} (-e\vb{p}\vdot\vb{E})\,.
\end{equation}

The relativistic kinetic energy is given by \(E_{kin}=(\gamma-1)m_e c^2\) and
thus its time derivative can be easily computed with the aid of
\cref{eq:gamma-time-derivative}
\[
\dv{E_{kin}}{t} = m_e c^2 \dv{\gamma}{t} =
m_e c^2 \frac{1}{\gamma} \frac{1}{{(m_e c)}^2} (-e\vb{p}\vdot\vb{E}) =
-\frac{1}{\gamma m_e} e \vb{p}\vdot\vb{E}
\]
or
\begin{equation}
  \label{eq:e-kin-deriv}
  \dv{E_{kin}}{t} = -e \vb{v}\vdot\vb{E}\,.
\end{equation}


We will now the equation of motion for the electron and we will begin with the
particular case of the \(O_y\) component of the motion.
Let us now consider the following
\[
\begin{aligned}
  \vb{E} &= -\pdv{\vb{A}}{t} = -\dv{\vb{A}}{t} +
    \underbrace{\pdv{\vb{A}}{\vb{x}}\pdv{\vb{x}}{t}}_{\left(\vb{v}\vdot\vb*{\nabla}\right)\vb{A}} \\
  \vb{v}\cp\vb{B} &= \vb{v} \cp (\curl{\vb{A}}) = \grad{(\vb{v}\vdot\vb{A})} -
    (\vb{v}\vdot\vb*{\nabla})\vb{A}\,,
\end{aligned}
\]
then \cref{eq:relativistic-lorentz} becomes
\[
\dv{\vb{p}}{t} = -e \left[-\dv{\vb{A}}{t}+(\vb{v}\vdot\vb*{\nabla})\vb{A}+
  \grad{(\vb{v}\vdot\vb{A})}-(\vb{v}\vdot\vb*{\nabla})\vb{A}\right]=
e\dv{\vb{A}}{t}-e\grad{(\vb{v}\vdot\vb{A})}\,.
\]

With the considerations in \cref{eq:simple-laser-A},
\(
\vb{A} = \vb{e}_y A_0 \sin{\phi}\,,
\)
where \(\phi=kx-\omega t\), the \(\vb{v}\vdot\vb{A}\) term becomes
\[
\vb{v}\vdot\vb{A} = v_y A_0 \sin{\phi}\,.
\]
Since we are dealing with a plane wave, the velocity and the vector potential
are independent and thus
\[
\pdv{y}(v_y A_0) = \pdv{z}(v_y A_0) = 0\,.
\]

The equation of motion for the electron for the \(O_y\) direction is then given by
\[
\dv{p_y}{t} = e \dv{A}{t}
\]
with the solution
\begin{equation}
  \label{eq:py-C1}
  p_y - eA = \mathcal{C}_1\,.
\end{equation}

We can observe that \(\mathcal{C}_1\) is a constant of the motion. In the case where
the electron is initially at rest, \(p_y=eA\).

Having emphasized the existence of the constants of the motion \(\mathcal{C}_1\),
we now move on to the complete equation of motion, which reads
\[
\dv{\tilde{\vb{p}}}{t} = \dv{t}
\begin{pmatrix}
  \tilde{p}_x \\
  \tilde{p}_y \\
  \tilde{p}_z
\end{pmatrix} =
- \frac{e}{m_e c} \left[
\begin{pmatrix}
  0 \\ E_0 \\ 0
\end{pmatrix}
+
\begin{pmatrix}
  v_x \\ v_y \\ v_z
\end{pmatrix}
\cp
\begin{pmatrix}
  0 \\ 0 \\ B_0
\end{pmatrix}
\right] \cos{\phi}\,,
\]
where we have used the notation \(\tilde{p} \equiv \frac{p}{m_e c}\). The above
equation can be further expanded into
\begin{equation}
\label{eq:normalized-lorentz}
\begin{aligned}
  \dv{\tilde{\vb{p}}}{t} &= - \frac{e}{m_e c} \left[
  \begin{pmatrix}
    0 \\ E_0 \\0
  \end{pmatrix}
  +
  \begin{pmatrix}
    \phantom{-}v_y B_0 \\
    -v_x B_0 \\
    0
  \end{pmatrix}
  \right] \cos{\phi} \\ &=
  - \frac{e E_0}{m_e c} \left[
  \begin{pmatrix}
    0 \\ 1 \\ 0
  \end{pmatrix}
  +
  \begin{pmatrix}
    \phantom{-}v_y / c \\
    -v_x/c \\
    0
  \end{pmatrix}
  \right] \cos{\phi} \\ &=
  -a_0\omega
  \begin{pmatrix}
    v_y / c \\
    1-v_x/c \\
    0
  \end{pmatrix}
  \cos{\phi}\,.
\end{aligned}
\end{equation}

Another constant of the motion can be found by considering the energy
equation~\eqref{eq:e-kin-deriv}. The time derivative of the \(\gamma\) factor
yields
\[
\begin{aligned}
  \dv{\gamma}{t} &= -\frac{e}{m_e c^2} \vb{v} \vdot
  \begin{pmatrix}
    0 \\ E_0 \\ 0
  \end{pmatrix}
  \cos{\phi} \\
  &= -\frac{e E_0}{m_e c}\frac{v_y}{c}\cos{\phi}
  = -a_0\omega \frac{v_y}{c}\cos{\phi}\,
\end{aligned}
\]
and we can thus express it using the normalised vector potential \(a_0\).

Taking a look at the \(O_x\) direction component of the equation of motion and
the \(O_y\) direction component of the energy equation shows some similarities
\[
\dv{\tilde{p}_x}{t} = -a_0\omega \frac{v_y}{c}\cos{\phi} = \dv{\gamma}{t}\,.
\]

Thus
\begin{equation}
  \label{eq:px-C2}
  \dv{t}(\tilde{p}_x - \gamma) = 0 \implies \gamma - \tilde{p}_x = \mathcal{C}_2
\end{equation}
we find a second constant of the motion \(\mathcal{C}_2\), whose values depend on
the initial conditions. Considering the electron at rest at \(t=0,x=0\), then
\(\vb{p}=0\), \(\gamma=1\) and \(\mathcal{C}_2=1\). With this result we can see
that at other times \(t\) that \(\gamma=1+\tilde{p}_x\). We observe that the
square of the \(\gamma\) factor can be written both as
\[
\gamma^2 = {\left(\sqrt{1+{\left(1+\frac{p}{m_e c}\right)}^2}\right)}^2 =
1 + {\left(1+\frac{p}{m_e c}\right)}^2  = 1 + \tilde{\vb{p}}^2
\]
and as
\[
\gamma^2 = {(1+\tilde{p}_x)}^2\,.
\]

Moreover, since from \cref{eq:normalized-lorentz} \(p_z=0\) for all times
\[
\begin{aligned}
  1 + \tilde{p}_x^2 + \tilde{p}_y^2 + \underbrace{\tilde{p}_z^2}_0 &=
  1 + 2\tilde{p}_x + \tilde{p}_x^2 \\
  \tilde{p}_y^2 &= 2\tilde{p}_x
\end{aligned}
\]
and thus we obtain a relation between forward momenta and transverse momenta
\begin{equation}
  \label{eq:px-py}
  \tilde{p}_x = \frac{1}{2} \tilde{p}_y^2\,.
\end{equation}

Taking into account \cref{eq:py-C1,eq:px-C2,eq:px-py}, the kinetic energy can be
written as
\[
E_{kin} = (\gamma-1)m_e c^2 = (1+\tilde{p}_x-1)m_e c^2 =
\frac{1}{2}\tilde{p}_y^2 m_e c^2 =
\frac{1}{2}\underbrace{{\left(\frac{eA}{m_e c}\right)}^2}_{a^2} m_e c^2\,,
\]
where the \(a\) notation was used similarly to the normalized vector potential
\(a_0\). With this notation, the kinetic energy relation has a formal similarity
with the classical one
\[
E_{kin} = m_e c^2 \frac{a^2}{2}
\]
considering that \(\tilde{p}_y=a\).

Before solving for the trajectory of the electron, let us first make some convenient
notations
\[
\tilde{p}_{x,y,z}=\frac{1}{m_e c}\gamma m_e v_{x,y,z} = \gamma \frac{v_{x,y,z}}{c}\,.
\]
Thus the momenta can be written as
\[
\tilde{\vb{p}} =
\begin{pmatrix}
  \tilde{p}_x \\
  \tilde{p}_y \\
  \tilde{p}_z
\end{pmatrix} =
\begin{pmatrix}
  \gamma v_x/c \\
  \gamma v_y/c \\
  0
\end{pmatrix} =
\begin{pmatrix}
  \frac{\gamma}{c}\dv{x}{t} \\
  \frac{\gamma}{c}\dv{y}{t} \\
  0
\end{pmatrix} =
\frac{\gamma}{c} \dv{t}
\begin{pmatrix}
  x \\ y \\ z
\end{pmatrix}\,.
\]
Since we have already found that the two constants of the motion are related
to the momenta, we can use those to find the trajectory of the electron
instead of directly solving \cref{eq:relativistic-lorentz}
\[
\begin{pmatrix}
  \tilde{p}_x \\
  \tilde{p}_y \\
  \tilde{p}_z
\end{pmatrix} =
\begin{pmatrix}
  \frac{a^2}2 \\
  a \\
  0
\end{pmatrix}\,.
\]

To further simplify calculations, we can move to the frame of reference with
\(\tau = t - \frac{x(t)}{c}\). In this case
\[
\gamma \dv{t} = \gamma \dv{\tau}{t}\dv{\tau} =
\gamma \left(1-\frac{1}{c}\dv{x}{t}\right)\dv{\tau} =
\bigg(\gamma-\underbrace{\frac{\gamma}{c}\dv{x}{t}}_{\tilde{p}_x}\bigg)\dv{\tau} =
\left(1+\tilde{p}_x - \tilde{p}_x\right)\dv{\tau} =
\dv{\tau}
\]
and thus the system of equations to be solved will be given by
\begin{equation}
\label{eq:trajectory-eq}
\begin{aligned}
  \dv{x}{\tau} &= c \frac{a^2}{2} \\
  \dv{y}{\tau} &= c a \\
  \dv{z}{\tau} &= 0\,.
\end{aligned}
\end{equation}

Since according to our notation
\[
a=\frac{e A}{m_e c}\,,
\]
we have
\[
\begin{aligned}
  \vb{a} &= \frac{e A_0}{m_e c}\vb{e}_y\sin{\phi} =
  a_0 \vb{e}_y \sin(kx -\omega t)\\ &=
  a_0 \vb{e}_y \sin(kx - \omega\tau - \omega\frac{x(t)}{c}) \\ &=
  a_0 \vb{e}_y \sin(-\omega \tau) \\ &=
 -a_0 \vb{e}_y \sin(\omega\tau)\,,
\end{aligned}
\]
where we have used the fact that \(k=\omega/c\).

Going now back to \cref{eq:trajectory-eq}, we can integrate to find the trajectory
of the electron. For the \(O_x\) component we have
\[
\begin{aligned}
  x(\tau) &= \frac{c}{2}\int a_0^2 \sin^2 \omega\tau \dd{\tau} =
  \frac{c a_0^2}{2}\int \left(1-\cos^2\omega\tau\right)\dd{\tau} =
  \frac{c a_0^2}{2}\int \frac{1-\cos(2\omega\tau)}{2}\dd{\tau} \\&=
  \frac{c a_0^2}{4}\left[\tau - \int \cos(2\omega\tau)\dd{\tau}\right] =
  \frac{c a_0^2}{4}\left[\tau - \frac{1}{2\omega}\sin(2\omega\tau)\right]\,.
\end{aligned}
\]
and for the \(O_y\) component
\[
y(\tau) = -c \int a_0 \sin(\omega\tau)\dd{\tau} = \frac{c a_0}{\omega}
\left[1 - \cos(\omega\tau)\right]\,.
\]

Thus, the motion of the electron in the selected frame of reference will be
given by
\[
\begin{aligned}
  x(\tau) &= \frac{c a_0^2}{4}\left[\tau - \frac{1}{2\omega}\sin(2\omega\tau)\right]\\
  y(\tau) &= \frac{c a_0}{\omega} \left[1 - \cos(\omega\tau)\right]\\
  z(\tau) &= 0\,.
\end{aligned}
\]

The trajectory is a simple oscillation on the \(O_y\) direction, but more complex
on the \(O_x\) direction. A plot illustrating the trajectory can be seen
in \cref{fig:electron-in-plane-wave}. Taking a closer look at the motion on the
\(O_x\) direction, we can see that it is a combination of a oscillation and
a drift motion. The drift motion is given by
\[
x = \frac{c a_0^2}{4} \tau = \frac{c a_0^2}{4} (t - \frac{x}{c}) =
\frac{c a_0^2}{4}t - \frac{a_0^2}{4}x
\]
or
\[
x = \frac{c a_0^2}{4} \frac{t}{1+\frac{a_0^2}{2}} = \frac{c a_0^2}{4+a_0^2}t
\]
with the drift velocity being given by
\[
v_{drift} = \frac{c a_0^2}{4+a_0^2}c
\]
in the laboratory frame. If we follow the motion of the electron in a reference
frame moving with the drift velocity, the motion of the electron will appear
as a figure-8, as can be seen in \cref{fig:electron-in-plane-wave}.

\begin{figure}[h]
  \centering
  \subimport{../figures/}{electron-in-plane-wave}%
  \caption{The trajectory of the electron in a plane wave in the laboratory frame
  (left) and in the average momentum frame (right)}\label{fig:electron-in-plane-wave}%
\end{figure}

\section{The Ponderomotive Force}

While the plane wave can be used as an approximation for a laser in some conditions,
it cannot be used when the laser is focused and has a significant spatial variation.
In the case of spatially varying fields, non-linear effects come into play and
give rise to new physics. One such phenomena is the ponderomotive force that
is excreted on charged particles in electromagnetic fields with non-vanishing
gradient.

Qualitatively, this force can be seen as a radiation pressure that
pushes charged particles away from the regions with high gradient. It is not
an instantaneous effect, but an averaged one and thus charges will have a
complicated motion, but on average they will be deflected due to this mechanism.
One important aspect of this force is that it is independent of the sign of the charge.
Thus, in a plasma, both electrons and nuclei are affected by this force, but
since electrons are much lighter, they will be deflected far away from their
equilibrium position.

We will begin with the derivation of the ponderomotive force in the non-relativistic
case, but this effect can also be obtained in a covariant fashion.

\subsection{Non-relativistic treatment}

In order to treat the nonlinear effects induced by a spatially varying field,
we will expand the electric field in a Taylor series.
Assuming that the electric field is separable in a spatially varying part and
a time dependent oscillation
\[
\vb{E}(\vb{r},t) = \vb{E}_s(\vb{r})\cos(\omega t)\,,
\]
then the spatial dependence of the electric field will be given by
\[
\vb{E}_s(\vb{r}) = \vb{E}_s(\vb{r})\eval_{\vb{r}=\vb{r}_0} +
\left(\Delta \vb{r}^{(1)}\vdot\vb*{\nabla}\right)\vb{E}_s(\vb{r})\eval_{\vb{r}=\vb{r}_0} + \order{\Delta\vb{r}^2}
\]
and the nonlinearity given by the magnetic field term can be expressed as a
function of the electric field by integrating
\[
\dv{\vb{B}}{t} = -\curl{\vb{E}}
\]
and thus
\[
\vb{B}^{(1)} = -\frac{1}{\omega}\curl{\vb{E}}\eval_{\vb{r}=\vb{r}_0} \sin(\omega t)\,,
\]
where we have used \textsuperscript{(n)} to denote the approximation order.

With these considerations, the nonlinear terms of the equation of motion yield
\[
m_e \dv{\vb{v}^{(1)}}{t} = -e \left[
\left(\Delta \vb{r}^{(1)}\vdot\vb*{\nabla}\right)\vb{E}(\vb{r},t)\eval_{\vb{r}=\vb{r}_0} +
\vb{v}^{(0)} \cp \vb{B}^{(1)}
\right]\,,
\]
with the linear terms corresponding to
\[
m_e \dv{\vb{v}^{(0)}}{t} = -e \vb{E}_s(\vb{r}_0)\cos(\omega t)\,.
\]

Solving the equation for the linear part, we obtain
\[
\begin{aligned}
  \vb{v}^{(0)} &= -\frac{e}{m_e \omega} \vb{E}_s \sin(\omega t) \\
  \Delta\vb{r}^{(1)} &= \frac{e}{m_e \omega^2} \vb{E}_s \cos(\omega t)\,.
\end{aligned}
\]
Replacing the results in the equation for the non-linear terms gives
\[
  m_e \dv{\vb{v}^{(1)}}{t} = -\frac{e^2}{m_e\omega^2} \left[
  \left(\vb{E}_s\vdot\vb*{\nabla}\right)\vb{E}_s(\vb{r},t)\eval_{\vb{r}=\vb{r}_0}\cos^2(\omega t) +
  \vb{E}_s \cp (\curl{\vb{E}}_s)\sin^2(\omega t)
  \right]
\]

Since the temporal and spatial parts are separated, we can get rid of the time
dependent part by averaging over a period. Since the time dependent terms are either
\(\sin^2\) or \(\cos^2\), when averaging over a period \(T=2\pi/\omega\)
\[
\left\langle m_e \dv{\vb{v}^{(1)}}{t}\right\rangle_T = \int_0^T m_e \dv{\vb{v}^{(1)}}{t} \dd{t}\,,
\]
we obtain a factor of \(1/2\) because
\(\left\langle\sin^2(\omega t)\right\rangle_T = \left\langle\cos^2(\omega t)\right\rangle_T\) and
\(\sin^2(x) + \cos^2(x)=1\).

Thus, we are left only with the spatial dependence
\begin{equation}
  \label{eq:non-linear-lorentz-avg}
  \left\langle m_e \dv{\vb{v}^{(1)}}{t}\right\rangle_T =
  -\frac{e^2}{2m_e\omega^2} \left[
  \left(\vb{E}_s\vdot\vb*{\nabla}\right)\vb{E}_s(\vb{r},t)\eval_{\vb{r}=\vb{r}_0} +
  \vb{E}_s \cp (\curl{\vb{E}}_s)
  \right]\,.
\end{equation}

The last term can be expanded into
\[
\begin{aligned}
  \vb{E}_s \cp (\curl{\vb{E}}_s) &=
  \vb{e}_i \varepsilon_{ijk}E_j\varepsilon_{klm}\partial_l E_m \\ &=
  \vb{e}_i \left(\delta_{il}\delta_{jm}-\delta_{im}\delta_{jl}\right)E_j\partial_l E_m =
  \vb{e}_i \left(E_j \partial_i E_j - E_j \partial_j E_i\right) \\ &=
  \vb{E}_s\vdot\grad{\vb{E}_s} - \left(\vb{E}_s\vdot\vb*{\nabla}\right)\vb{E}_s \\ &=
  \frac{1}{2}\grad(\vb{E}_s^2) - \left(\vb{E}_s\vdot\vb*{\nabla}\right)\vb{E}_s\,,
\end{aligned}
\]
where we have omitted the \textsubscript{s} subscript for the \(\vb{E}_s\) vector
components to simplify the notation. Using this result in \cref{eq:non-linear-lorentz-avg}
yields
\begin{equation}
  \label{eq:ponderomotive-force}
  \left\langle m_e \dv{\vb{v}^{(1)}}{t}\right\rangle_T =
  -\frac{e^2}{4m_e\omega^2} \grad(\vb{E}_s^2) \equiv \vb{F}_{pond}
\end{equation}
the equation for the ponderomotive force.

\subsection{Relativistic treatment}

The ponderomotive force can also be obtained with a more rigorous relativistic
treatment. As \Textcite{mulser_highpower_2010} show in Chapter 5, the ponderomotive
force can be derived in multiple ways and the treatment can also be extended from
one charged particle to many.

Thus for the relativistic case it can be shown that
\[
\vb{F}_{pond} \sim -\alpha \grad{(\vb{E}^2)}\,,
\]
where \(\alpha=q^2/4m\). In the relativistic case the sign of the ponderomotive
force can change depending on the phase velocity~\autocite[205]{mulser_highpower_2010}.

\section{Laser Wakefield Acceleration}

\Textcite{tajima_laserelectron_1979} proposed a novel method for accelerating
electrons using intense laser pulses.
When an intense laser pulse enters an under-dense plasma target,
it displaces the electrons far from their equilibrium positions
due to the ponderomotive force. The nuclei have a much larger
mass than the electrons and thus react on a larger time scale
and in a first approximation can be considered immobile.
The laser continues to propagate and the electrons will move
back to their equilibrium position, but they will overshoot.
Thus we will have an accumulation of charge behind the laser pulse
followed by a region that lacks electrons. In this structure
that forms behind the laser pulse and that travels along with
the pulse, an intense electric field is formed between the electrons
in the back and the region devoid of charge in the front, where
we have mostly positive charges from the ion and nuclei background.

If we can inject electrons in this ``bubble'' that travels along
with the pulse, than we can accelerate them. The advantage of this
method is that the plasma can sustain very large electric fields.
Classical particle accelerators have a limit on how much energy
can be transferred to the accelerated particles per metre of
accelerator. This limit is given by the material, since the
crystalline structure of the metals will break down at high enough
electric fields. Thus laser plasma accelerators can transfer much
more energy per metre of accelerator and thus they can be
significantly smaller than conventional accelerators.

Of course, plasma based accelerators are not without
limitations~\autocite[8]{oneil_laserwakefield_2017}.
The accelerated electrons must be synchronized with the bubble that
is formed behind the pulse and of course, the bubble must be stable.
When the electrons fall out of sync, phenomena which is called
electron dephasing, they will be decelerated or scattered.
Other limitations are given by how well is the laser pulse focused
and by the length on which the energy of the laser dissipates in
the plasma.
If the laser pulse is focused too fast, it will defoucs after a
similarly short time and the ponderomotive force acting on the
electrons will be less efficient at displacing them.

The nature of the plasma accelerators itself poses challenges
regarding the injection of particles to be accelerated and the
quality of the resulting beam.

\end{document}
