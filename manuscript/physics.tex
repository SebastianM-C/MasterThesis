\documentclass[12pt, class=report, crop=false]{standalone}
\usepackage{msc_thesis}

% !TeX spellcheck = en-GB
% !TEX bib = reference.bib
% chktex-file 21 # This command might not be intended.

\begin{document}

\chapter{Classical Electrodynamics}%
\label{chap:classical-electrodynamics}

In order to study complex phenomena such as laser wakefield acceleration,
we need to have a solid understanding of the basic physical phenomena
that govern the dynamics of charged particles in interaction
with electromagnetic fields. In this thesis we will restrict ourselves to
classical electrodynamics, ignoring QED effects that are important
for very high laser intensities \(I \gtrsim \SI{5e22}{\watt\per\square\centi\metre}\).
We will mainly follow the ideas presented in~\textcite{jackson_classicalelectrodynamics_1999} and \textcite[Chapter 2]{eisenberg_nucleartheory_1978}.

We thus begin with Maxwell's equations in free space
\begin{subequations}%
  \label{eq:maxwell}
  \begin{align}
    \div{\vb{E}} & = \frac{\rho}{\varepsilon_0} \label{eq:coulomb-law} \\
    \div{\vb{B}} & = 0 \label{eq:gauss-law} \\
    \curl{\vb{E}} & = - \pdv{\vb{B}}{t} \label{eq:faraday-law} \\
    \curl{\vb{B}} & = \mu_0 \vb{j} + \frac{1}{c^2} \pdv{\vb{E}}{t} \,, \label{eq:ampere-law}
  \end{align}
\end{subequations}

which relate the electromagnetic field to sources. An additional
equation must be satisfied in order to ensure charge conservation
\begin{equation}
  \label{eq:continuity-equation}
  \div{\vb{j}(\vb{r}, t)} + \pdv{\rho(\vb{r}, t)}{t} = 0 \,.
\end{equation}

As we can see above, \cref{eq:faraday-law,eq:gauss-law}
do not involve sources and thus they state the dynamical properties of the fields.
Since \cref{eq:coulomb-law,eq:ampere-law} describe how
the sources influence the fields, we need an additional equation to describe how
the fields affect the sources
\[
  \vb{F} = \int \dd{\vb{r}'} \rho(\vb{r}', t) \vb{E}(\vb{r}', t) +
           \frac{1}{c} \int \dd{\vb{r}'} \vb{j}(\vb{r}', t) \cross \vb{B}(\vb{r}', t)\,.
\]

Maxwell's equations~\eqref{eq:maxwell} relate six field quantities (\(\vb{E}\) and \(\vb{B}\))
to four source quantities (\(\rho\) and \(\vb{j}\)). This implies that there are some
restrictions on the six quantities. This suggests that we can find a less redundant
way to express the fields, and indeed the four quantities given by the
vector potential \(\vb{A}\) and scalar potential \(\rho\) provide this representation.
\Cref{eq:gauss-law} implies the existence of a vector potential
\begin{equation}
  \label{eq:vector-potential}
  \vb{B}(\vb{r}, t) = \curl{\vb{A}(\vb{r}, t)}\,.
\end{equation}

Substituting~\eqref{eq:vector-potential} in~\eqref{eq:faraday-law} we obtain
\begin{equation}
  \label{eq:faraday-vector-potential}
  \curl(\vb{E} + \pdv{\vb{A}}{t}) = 0
\end{equation}

and thus the quantity in the parenthesis can always be expressed as the
gradient of a scalar field, namely the scalar potential
\[
\grad{\phi}(\vb{r}, t) = -\vb{E}(\vb{r}, t) - \pdv{\vb{A}}{t}\,.
\]

With these considerations \cref{eq:coulomb-law} becomes
\[
  \div(\grad{\phi} + \pdv{\vb{A}}{t}) = - \frac{\rho}{\varepsilon_0}
\]
or
\begin{equation}
  \label{eq:coulomb-scalar-potential}
  \laplacian{\phi} + \pdv{t}\div{\vb{A}} = - \frac{\rho}{\varepsilon_0}
\end{equation}

and \cref{eq:ampere-law}

\begin{equation}
  \label{eq:ampere-potentials-pre}
  \curl(\curl{\vb{A}}) = \mu_0 \vb{j}
    - \frac{1}{c^2} \pdv{t} \left( \grad{\phi} + \pdv{\vb{A}}{t} \right)\,.
\end{equation}

Using the following vector identity
\begin{equation}
  \label{eq:curl-of-curl}
  \curl(\curl{\vb{A}}) = \grad(\div{\vb{A}}) - \laplacian{\vb{A}}\,,
\end{equation}

\cref{eq:ampere-potentials-pre} becomes
\[
  \grad(\div{\vb{A}}) - \laplacian{\vb{A}} = \mu_0 \vb{j}
    - \frac{1}{c^2} \left( \grad{\pdv{\phi}{t}} + \pdv[2]{\vb{A}}{t} \right)
\]
or
\begin{equation}
  \label{eq:ampere-potentials}
  \laplacian{\vb{A}} - \frac{1}{c^2}\pdv[2]{\vb{A}}{t} =
    -\mu_0 \vb{j} + \grad(\div{\vb{A}} + \frac{1}{c^2} \pdv{\phi}{t})\,.
\end{equation}

\Cref{eq:coulomb-scalar-potential,eq:ampere-potentials}
were obtained by substituting the potentials obtained from the source-less
equations,~\eqref{eq:gauss-law} and~\eqref{eq:faraday-law}, into the ones
with sources,~\eqref{eq:coulomb-law} and~\eqref{eq:ampere-law}. They are thus
fully equivalent with Maxwell's equations~\eqref{eq:maxwell} and, as we can observe,
relate the four quantities given by the potentials to the four quantities for the
sources. They also preserve the invariance under Lorentz transformations, with
the scalar potential \(\phi\) as the time-like component.

\Cref{eq:coulomb-scalar-potential,eq:ampere-potentials}
can be simplified by decoupling the potentials. This is possible due to the fact
that potentials are not unique. To illustrate this point consider
\[
  \vb{A}'(\vb{r},t) = \vb{A}(\vb{r},t) + \grad{\Lambda}(\vb{r},t)\,.
\]

This vector potential gives rise to a magnetic field
\[
  \curl{\vb{A}'} = \curl{\vb{A}} + \curl(\grad{\Lambda}) = \curl{\vb{A}} = \vb{B}
\]
equal with the original one since \( \curl(\grad{\varphi}) = 0 \).

Similarly, for a scalar potential
\[
  \phi'(\vb{r},t) = \phi(\vb{r},t) - \pdv{\Lambda(\vb{r},t)}{t}
\]
and the corresponding electric field will be
\[
  - \grad{\phi'} - \pdv{\vb{A'}}{t} =
  - \grad{\phi} + \grad{\pdv{\Lambda}{t}} - \pdv{\vb{A}}{t} - \pdv{t} \grad{\Lambda}
  = - \grad{\phi} - \pdv{\vb{A}}{t}
  = \vb{E}\,,
\]
since the spatial and temporal derivatives commute. These kinds of transformations
are called gauge transformations.

\section{Gauge transformations}

The freedom of choosing the gauge leads to the following condition satisfied by
the scalar and vector potentials
\[
  \div{\vb{A}} + \frac{1}{c^2}\pdv{\phi}{t} = 0\,,
\]
called the Lorenz condition.

Indeed, if we consider a set of potentials \(\vb{A}\) and \(\phi\) that
don't satisfy the condition
\[
  \div{\vb{A}} + \frac{1}{c^2}\pdv{\phi}{t} \ne 0 = f(\vb{r},t)\,,
\]
then we can always carry out a gauge transformation to a new set of potentials
\(\vb{A}'\) and \(\phi'\) that satisfy the Lorenz condition, such that
\begin{align*}
  \div{\vb{A}} + \frac{1}{c^2}\pdv{\phi}{t} &= \div(\vb{A}' - \grad{\Lambda})
    + \frac{1}{c^2}\pdv{t} \left( \phi' + \pdv{\Lambda}{t} \right) \\
    &= \div{\vb{A}'} - \laplacian{\Lambda} + \frac{1}{c^2}\pdv{\phi'}{t}
    + \frac{1}{c^2}\pdv[2]{\Lambda}{t}
    = f(\vb{r},t)
\end{align*}

or
\[
  \div{\vb{A}} + \frac{1}{c^2}\pdv{\phi}{t} =
  \dalambert \Lambda \equiv \frac{1}{c^2}\pdv[2]{\Lambda}{t} - \laplacian{\Lambda} = f(\vb{r},t)\,,
\]

where the d'Alambertian operator is defined as
\[
  \dalambert \equiv \frac{1}{c^2}\pdv[2]{t} - \laplacian
\]
when choosing the Minkowski metric \( (+,-,-,-) \) and
\[
  \div{\vb{A'}} + \frac{1}{c^2}\pdv{\phi'}{t} = 0\,,
\]
since they satisfy the Lorenz condition.
The transformation we need is thus defined by the solution of \(\dalambert \Lambda = f\).

Imposing the Lorenz condition on equations~\eqref{eq:coulomb-scalar-potential}
and~\eqref{eq:faraday-vector-potential} decouples the potentials
\begin{align*}
  \laplacian{\phi} - \pdv{t} \frac{1}{c^2} \pdv{\phi}{t} = -\frac{\rho}{\varepsilon_0} \\
  \laplacian{\vb{A}} - \frac{1}{c^2}\pdv[2]{\vb{A}}{t} = -\mu_0 \vb{j}
\end{align*}
yielding the simplified form of Maxwell's equations
\begin{align*}
  \dalambert \phi &= \frac{\rho}{\varepsilon_0} \\
  \dalambert \vb{A} &= \mu_0 \vb{j}\,.
\end{align*}

This form of Maxwell's equations preserves Lorentz invariance, as the Lorenz
gauge condition can be expressed in a covariant way as the contraction of the
four-vector \(A \equiv (\frac{\phi}{c}, \vb{A})\) with the four-gradient
\((\frac{1}{c}\pdv{t}, -\grad)\).

Since the Lorenz condition doesn't fix the gauge, but only restricts us to
transformations with \(\dalambert \Lambda = 0\), we can impose further conditions
in order to fix the gauge, but in general those will not be covariant.
One such condition is given by the Coulomb gauge
\begin{equation}
  \label{eq:coulomb-gauge}
  \div{\vb{A}} = 0.
\end{equation}

In this gauge \cref{eq:coulomb-scalar-potential} becomes a Poisson equation
for the scalar potential
\begin{equation}
  \label{eq:coulomb-poisson}
  \laplacian{\phi} = -\frac{\rho}{\varepsilon_0}
\end{equation}
with the solution given by the instantaneous Coulomb potential of the charge
density in the domain \(\rho(\vb{r},t)\)
\begin{equation}
  \label{eq:scalar-potential-solution}
  \phi(\vb{r},t) = \frac{1}{4\pi \varepsilon_0} \int \frac{\rho(\vb{r}',t)}{|\vb{r}-\vb{r}'|} \dd{\vb{r}}'\,,
\end{equation}
explaining the name of the condition~\eqref{eq:coulomb-gauge}.

An apparent violation of special relativity shows up in the above result which
states that the scalar potential (at time \(t\)) is given by the instantaneous Coulomb
interactions between charges (also at time \(t\)). The contradiction is only
apparent and stems from the act that the Coulomb gauge is not Lorentz invariant.

In order to resolve the contradiction we first note that we can only observe
the electric field
\[
  \vb{E}(\vb{r}, t) = -\grad{\phi}(\vb{r},t) -\pdv{\vb{A}(\vb{r},t)}{t}\,.
\]
Thus, the instantaneous propagation is removed by the time derivative
of the vector potential.

In the Coulomb gauge, the vector potential is given by
\begin{equation}
  \label{eq:vector-potential-coulomb-gauge}
  \dalambert \vb{A} = \mu_0 \vb{j} - \frac{1}{c^2} \grad{\pdv{\phi}{t}}\,.
\end{equation}

Considering the continuity equation~\eqref{eq:continuity-equation} and
the form of the scalar potential in \cref{eq:scalar-potential-solution},
the second term in \cref{eq:vector-potential-coulomb-gauge} becomes
\begin{equation}
  \label{eq:scalar-potential-continuity-equation}
  \grad{\pdv{\phi}{t}} = \grad \frac{1}{4\pi \varepsilon_0}
    \int \frac{\pdv{\rho}{t}}{|\vb{r}-\vb{r}'|} \dd{\vb{r}}'
    = - \frac{1}{4\pi \varepsilon_0} \grad
    \int \frac{\boldsymbol{\nabla}' \vdot \vb{j}(\vb{r}',t)}{|\vb{r}-\vb{r}'|} \dd{\vb{r}}'\,,
\end{equation}
where \(\boldsymbol{\nabla}'\) denotes the derivatives with respect to \(\vb{r}'\).
Using the Helmholtz decomposition we can write any sufficiently well behaved
vector (the current density in this particular case) as the sum
of a divergence-free (transversal) component and a curl-free (longitudinal) one:
\[
  \vb{j} = \vb{j}^t + \vb{j}^l \,,
\]
where
\begin{align*}
  \div{\vb{j}^t} &= 0 \\
  \curl{\vb{j}^l} &= 0 \,.
\end{align*}

Using the vector identity~\eqref{eq:curl-of-curl} and
\[
  \laplacian{\frac{1}{|\vb{r}-\vb{r}'|}} = -4\pi \delta(\vb{r}-\vb{r}')
\]
we can write the current density as follows
\begin{align*}
  \laplacian(\vb{j}^t + \vb{j}^l) &= \grad(\div{\vb{j}^l}) - \curl(\curl{\vb{j}^t}) \\
  \int \frac{\laplacian{\vb{j}}}{|\vb{r}-\vb{r}'|} \dd{\vb{r}} &=
    \int \frac{\grad(\div{\vb{j}^l})}{|\vb{r}-\vb{r}'|} \dd{\vb{r}'}
    - \int \frac{\curl(\curl{\vb{j}^t})}{|\vb{r}-\vb{r}'|} \dd{\vb{r}'} \\
  -4\pi \vb{j} &= \grad \int \frac{\div{\vb{j}^l}}{|\vb{r}-\vb{r}'|} \dd{\vb{r}'}
    - \curl \curl \int \frac{\vb{j^t}}{|\vb{r}-\vb{r}'|} \dd{\vb{r}'}
\end{align*}
and thus we obtain the two components as

\begin{align*}
  \vb{j}^t &= \frac{1}{4\pi} \curl \curl \int \frac{\vb{j}(\vb{r}',t)}{|\vb{r}-\vb{r}'|} \dd{\vb{r}}' \\
  \vb{j}^l &= -\frac{1}{4\pi} \grad \int
    \frac{\boldsymbol{\nabla}' \vdot \vb{j}(\vb{r}',t)}{|\vb{r}-\vb{r}'|} \dd{\vb{r}}'\,.
\end{align*}

Comparing with \cref{eq:scalar-potential-continuity-equation} we see that
\[
  \frac{1}{c^2} \grad{\pdv{\phi}{t}} = \frac{\varepsilon_0}{c^2} \vb{j}^l
  = \mu_0 \vb{j}^l
\]
and thus the source term in \cref{eq:vector-potential-coulomb-gauge} can
be expressed as function of the transverse current:
\[
  \dalambert \vb{A} = \mu_0 (\vb{j} - \vb{j}^l) = \mu_0 \vb{j}^t\,
\]
and this also why the Coulomb gauge is also called the transverse gauge.
This gauge is useful when no sources are present. In this case \(\phi=0\),
\(\vb{A}\) satisfies the homogeneous wave equation and the fields can
be expressed only as function of the vector potential
\begin{align*}
  \vb{E} &= -\pdv{\vb{A}}{t} \\
  \vb{B} &= \curl{\vb{A}}\,.
\end{align*}

\section{The Poynting theorem}

In order to complete the description of the interaction between fields
and sources, we will now focus on how the fields affect the particles.
We begin by considering the force acting on a charge \(q\)
\[
  \vb{F} = q\vb{E} + q\vb{v} \cp \vb{B}\,.
\]

The corresponding infinitezimal variation of the force is given by
\[
  \var{\vb{F}} = \rho \vb{E} \var{V} + \vb{j} \cp \vb{B} \var{V}
    = (\rho \vb{E} + \vb{j} \cp \vb{B}) \var{V}
    \equiv f \var{V},
\]
where \(f = \rho \vb{E} + \vb{j} \cp \vb{B}\) is the Lorentz force
density. We can now consider a uniform charge distribution characterized
by \(\rho\). For an infinitezimal volume \(\var{V}\) of this charge distribution,
the rate of change of the work, or the power given by the
fields is given by
\[
  \vb{v} \vdot \vb{F} = \rho \vb{v} \vb{E} + \frac{\vb{j}}{q} \vdot (\vb{j} \cp \vb{B})
    = \rho \vb{v} \vb{E}\,.
\]

As we can see above, the magnetic force doesn't contribute to the work
done by the fields. Thus, the power transferred from the fields to the
charges in a finite domain \(\mathscr{D}\) is
\[
  \int_{\mathscr{D}} \vb{j} \vdot \vb{E} \dd{\vb{r}}\,.
\]

For the energy to conserve, this power must be balanced by a corresponding
rate of decrease of energy in the electromagnetic field.
Using the Ampère law~\eqref{eq:ampere-law}
\[
  \int_{\mathscr{D}} \vb{j} \vdot \vb{E} \dd{\vb{r}} =
  \int_{\mathscr{D}} \vb{E} \vdot \frac{1}{\mu_0} \qty(\curl{\vb{B}} - \frac{1}{c^2} \pdv{\vb{E}}{t}) \dd{\vb{r}} =
  \frac{1}{\mu_0} \int_{\mathscr{D}} \left[\vb{E} \vdot(\curl{\vb{B}}) -
    \frac{1}{c^2} \vb{E} \vdot \pdv{\vb{E}}{t} \right] \dd{\vb{r}}
\]

Using the following vector identity
\[
  \div(\vb{E} \cp \vb{B}) = \vb{B} \vdot (\curl{E}) - \vb{E} \vdot (\curl{\vb{B}})
\]
we can express \(\vb{E} \vdot(\curl{\vb{B}})\) as
\[
  \vb{E} \vdot(\curl{\vb{B}}) = \vb{B} \vdot (\curl{E}) - \div(\vb{E} \cp \vb{B}) =
  - \vb{B} \vdot \pdv{\vb{B}}{t} - \div(\vb{E} \cp \vb{B})\,,
\]
where we used \cref{eq:faraday-law} for the first term.

Using this result, the power transferred by the fields is given by
\[
  \int_{\mathscr{D}} \vb{j} \vdot \vb{E} \dd{\vb{r}} =
  -\int_{\mathscr{D}} \left[\frac{1}{\mu_0} \div(\vb{E} \cp \vb{B}) + \frac{1}{\mu_0} \vb{B} \vdot \pdv{\vb{B}}{t} +
    \frac{1}{\mu_0 c^2} \vb{E} \vdot \pdv{\vb{E}}{t} \right] \dd{\vb{r}}
\]

Considering that
\[
  \vb{E} \vdot \pdv{\vb{E}}{t} = \frac{1}{2} \pdv{t} \vb{E}^2\,,
\]
we obtain
\[
  \int_{\mathscr{D}} \vb{j} \vdot \vb{E} \dd{\vb{r}} =
  -\int_{\mathscr{D}} \left[ \frac{1}{2} \pdv{t}
    \left( \varepsilon_0 \vb{E}^2 + \frac{1}{\mu_0}\vb{B}^2 \right)
  + \frac{1}{\mu_0}\div(\vb{E} \cp \vb{B}) \right] \dd{\vb{r}}\,.
\]

The total energy density of the electromagnetic field can be denoted with
\[
  w_{em} = \frac{1}{2} \left( \varepsilon_0 \vb{E}^2 + \frac{1}{\mu_0}\vb{B}^2 \right)
\]
and thus we obtain
\[
  -\int_{\mathscr{D}} \vb{j} \vdot \vb{E} \dd{\vb{r}} =
  \int_{\mathscr{D}} \left[ \pdv{w_{em}}{t} +
  \frac{1}{\mu_0}\div(\vb{E} \cp \vb{B}) \right] \dd{\vb{r}}\,.
\]

Since the domain \(\mathscr{D}\) is arbitrary, we can write the above as a
differential continuity equation
\begin{equation}
  \label{eq:poynting-theorem-local}
  \pdv{w_{em}}{t} = - \div{\vb{S}} - \vb{j} \vdot \vb{E}\,,
\end{equation}
where
\[
  \vb{S} = \frac{1}{\mu_0} \vb{E} \cp \vb{B}
\]
is the Poynting vector representing the energy flow.

If we consider the domain \(\mathscr{D}\) such that no particles will leave it
\[
  W_{em} = \int_{\mathscr{D}} w_{em} \dd{\vb{r}}
\]
is the energy of the electromagnetic field and \(W_{mech}\) is the
energy of the particles
\[
  W_{mech} = \int_{\mathscr{D}} w_{mech} \dd{\vb{r}} =
  \int_{\mathscr{D}} \vb{j} \vdot \vb{E} \dd{\vb{r}}\,.
\]

By using Gauss' theorem, the energy flux corresponding to the
Poynting vector becomes
\[
  \int_{\mathscr{D}} \div{\vb{S}} = \oint_\Sigma \vb{n} \vdot \vb{S} \dd{a}\,,
\]
where \(\Sigma\) is the surface enclosing the domain \(\mathscr{D}\).

With the above considerations Poynting's theorem gives the conservation of energy for the whole system
\begin{equation}
  \label{eq:poynting-theorem}
  \dv{W}{t} = \dv{t} \left( W_{em} + W_{mech} \right) =
  -\oint_\Sigma \vb{n} \vdot \vb{S} \dd{a}\,,
\end{equation}

stating that the rate of change of the energy of the system composed of the charged
particles and corresponding fields is given by minus the flux of the
Poynting vector through the surface bounding the domain.

\Cref{eq:poynting-theorem-local} is the local form for the Poynting
theorem.

If we consider the extension of the domain to infinity \(\mathscr{D} \to \mathbb{R}^3\),
\(\Sigma \to \Sigma_\infty\), then there is no energy flow through the
boundary since electromagnetic waves propagate at a constant finite speed \(c\). Then
\[
  \dv{W}{t} = \dv{t} \left( W_{em} + W_{mech} \right) = 0
\]
and the entire energy of the electromagnetic field can be converted into
the mechanical energy of the particles interacting with the field.

If we consider \(\mathscr{D}\) such that it doesn't enclose any sources, than
\[
  \dv{t} W_{em} = -\oint_\Sigma \vb{n} \vdot \vb{S} \dd{a}\,,
\]
which shows that the energy of the electromagnetic field in the domain
\(\mathscr{D}\) can change through the variation of the flux of the
Poynting vector on the boundary of the domain, \(\Sigma\). Thus we can
indeed say that the flux of the Poynting vector is the energy flux.

\section{Electromagnetic waves}

In the of sources, Maxwell's equations become
\begin{subequations}
  \begin{align*}
    \div{\vb{B}} &= 0  \qquad \curl{\vb{E}} + \pdv{\vb{B}}{t} = 0 \\
    \div{\vb{D}} &= 0  \qquad \curl{\vb{H}} - \pdv{\vb{D}}{t} = 0\,,
  \end{align*}
\end{subequations}
where \(\vb{D} = \varepsilon \vb{E}\) and \(\vb{B} = \mu \vb{H}\).
In our case, \(\varepsilon = \varepsilon_0\) and \(\mu = \mu_0\).

If we assume that the time dependence for the solutions is given by
\(\ee^{-\ii \omega t}\), then the above equations become
\begin{subequations}
  \begin{align*}
    \div{\vb{B}} &= 0  \qquad \curl{\vb{E}}\ee^{-\ii \omega t} - \ii \omega \vb{B} \ee^{-\ii \omega t} = 0 \\
    \div{\vb{D}} &= 0  \qquad \curl{\vb{H}}\ee^{-\ii \omega t} + \ii \omega \vb{D} = 0\,.
  \end{align*}
\end{subequations}
More complex time dependencies can be treated with a Fourier decomposition since if we have a solution, any linear combinations
with that solution are also solutions.

If we consider only the curl equations
\[
\begin{aligned}
  \curl{\vb{E}} - \ii \omega \vb{B} = 0 \\
  \curl{\vb{B}} + \ii \omega \mu \epsilon \vb{E} = 0
\end{aligned}
\]
and take the curl
\[
\curl{\curl{\vb{E}}} - \ii \omega \curl{\vb{B}} = \grad{\underbrace{(\div{\vb{E}})}_{0}}
- \laplacian{\vb{E}} + {(\ii \omega)}^2 \mu \varepsilon \vb{E} = 0
\]
we obtain the Helmholtz wave equations
\[
\begin{aligned}
  \left(\laplacian + \omega^2 \mu \varepsilon \right)\vb{E} &= 0 \\
  \left(\laplacian + \omega^2 \mu \varepsilon \right)\vb{B} &= 0
\end{aligned}
\]
or in a more compact notation
\begin{equation*}
\label{eq:helmholtz-wave-eq}
\left(\laplacian + \omega^2 \mu \varepsilon \right)
\begin{pmatrix}
  \vb{E} \\ \vb{B}
\end{pmatrix}
= 0\,.
\end{equation*}

Taking the plane wave
\[
\ee^{\ii k x - \ii \omega t}
\]
as a solution for the Helmholtz wave equation~\eqref{eq:helmholtz-wave-eq}, we can study
what properties arise for wave number \(k\) and the frequency \(\omega\).
Thus we can see that
\[
-k^2 + \omega^2 \mu \varepsilon = 0
\]
or
\[
k = \omega \sqrt{\mu \varepsilon}\,.
\]

The phase velocity of a wave \(v_\phi\) is defined as
\[
v_\phi = \frac{\omega}{k}\,.
\]

In this case, the phase velocity of a plane wave satisfying the
Helmholtz equation is
\[
v_\phi = \frac{1}{\sqrt{\mu\varepsilon}} = \frac{c}{n}\,,
\]
where \(n\) is the refraction index of the media in which the wave propagates and is given by
\[
n = \sqrt{\frac{\mu}{\mu_0}\frac{\varepsilon}{\varepsilon_0}}\,.
\]

A general solution for the wave equation can be constructed by
using the superposition principle
\[
u(x,t) = a \ee^{\ii k x - \ii \omega t} + b \ee^{-\ii k x - \ii \omega t}\,.
\]
This general solution can be seen as a superposition of incoming and outgoing plane waves.

Let us now consider an electromagnetic plane wave, that is a
plane wave that satisfies both the Helmholtz wave equation~\eqref{eq:helmholtz-wave-eq}
and Maxwell's equations~\eqref{eq:maxwell}
\[
\begin{aligned}
  \vb{E}(\vb{x},t) &= \vb*{\mathcal{E}} \ee^{\ii k \vb{n} \vdot \vb{x} - \ii \omega t} \\
  \vb{B}(\vb{x},t) &= \vb*{\mathcal{B}} \ee^{\ii k \vb{n} \vdot \vb{x} - \ii \omega t}\,.
\end{aligned}
\]

First, let us consider the Laplacian in the Helmholtz equation
\[
\pdv[2]{x_i} \mathcal{E}_i \ee^{\ii k n_j x_j - \ii \omega t} =
\pdv{x_i} \ii k n_j \delta_{ij} \mathcal{E}_i \ee^{\ii k n_j x_j - \ii \omega t} =
-k^2 n_i n_i \mathcal{E}_i \ee^{\ii k n_j x_j - \ii \omega t}\,.
\]

With this consideration, Helmholtz equation yields
\[
-k^2 n_i n_i \mathcal{E}_i \ee^{\ii k n_j x_j - \ii \omega t}
+ \mu\varepsilon\omega^2\mathcal{E}_i \ee^{\ii k n_j x_j - \ii \omega t} = 0
\]
or
\[
k^2 \vb{n} \vdot \vb{n} = \mu\varepsilon\omega^2\,.
\]

Considering that \(k=\sqrt{\mu\varepsilon} \omega\), we obtain
that the norm of \(\vb{n}\) must be 1
\[
\vb{n}\vdot\vb{n}=1\,.
\]

Now we continue with Maxwell's equations. We begin with the divergence
equations
\[
\div{\vb{B}} = 0 \implies \partial_i \mathcal{B}_i \ee^{\ii k n_j x_j - \ii \omega t} = 0\,,
\]
where we have used the notation
\[
\partial_i \equiv \pdv{x_i}\,.
\]
Thus
\(
\ii k n_i \mathcal{B}_i = 0
\)
and the magnetic field \(\vb{B}\) and \(\vb{n}\) are perpendicular since
\[
\vb{n} \vdot \vb*{\mathcal{B}} = 0\,.
\]

Similarly, for the electric field
\[
\div{\vb{D}} = 0 \implies \partial_i \varepsilon \mathcal{E}_i \ee^{\ii k n_j x_j - \ii \omega t} = 0
\]
and
\[
\ii k n_i \mathcal{E}_i = 0 \implies \vb{n} \vdot \vb*{\mathcal{E}}=0\,.
\]
Thus, since the electric and magnetic fields are perpendicular to
the propagation direction given by the \(\vb{n}\) versor, the
electromagnetic plane wave is a transverse wave.

If we now consider the curl equation
\[
\curl{\vb{E}} = -\pdv{\vb{B}}{t}\,,
\]
we obtain
\[
\begin{aligned}
  \varepsilon_{ijk} \partial_j E_k &= -(-\ii \omega) \mathcal{B}_i
  \ee^{\ii k n_i x_i- \ii \omega t} \\
  \varepsilon_{ijk} \partial_j \mathcal{E}_k \ee^{\ii k n_i x_i- \ii \omega t} &= \ii \omega \mathcal{B}_i
  \ee^{\ii k n_i x_i- \ii \omega t} \\
  \varepsilon_{ijk} \mathcal{E}_k \ii k n_j &= \ii \omega \mathcal{B}_i\,.
\end{aligned}
\]

Thus \(\vb{n} \cp \vb*{\mathcal{E}} \sqrt{\mu\varepsilon} = \vb*{\mathcal{B}}\) or
\[
\vb{n} \cp \vb*{\mathcal{E}} \frac{n}{c} = \vb*{\mathcal{B}}\,.
\]
In vacuum (\(n=1\)), we observe that the electric and magnetic field differ in magnitude by a factor of \(c\)
\[
\abs{\vb*{\mathcal{E}}} = \abs{c\vb*{\mathcal{B}}}\,.
\]

If we consider a coordinate system spanned by the versors
\((\vb*{\epsilon}_1,\vb*{\epsilon}_2,\vb{n})\), then the electric
and magnetic field magnitudes can be written as
\[
\vb*{\mathcal{E}} = \vb*{\epsilon}_1 E_0 \qquad
\vb*{\mathcal{B}} = \vb*{\epsilon}_2 \sqrt{\mu\varepsilon} E_0\,.
\]
A plane wave with its electric field always in a direction \(\vb*{\epsilon}_1\) is called \emph{linearly polarized} with the
polarization vector \(\vb*{\epsilon}_1\).

According to the Poynting theorem, the plane wave transports energy. The Poynting vector in this case is given by
\[
\vb{S} = \vb{E} \cp \vb{H} =
\vb{e}_i \varepsilon_{ijk} {\vb*{\epsilon}_1}_j E_0 {\vb*{\epsilon}_2}_k \frac{1}{\mu} \sqrt{\mu\varepsilon} =
\sqrt{\frac{\varepsilon}{\mu}} E_0^2\vb{n}
\]

\section{Electron in a Plane Wave}

In this section we will consider the classical dynamics of an electron in a
laser pulse following the discussion in~\textcite{karsch_applicationshigh_2018}.
The starting point is the equation of motion for the electron
\begin{equation}
  \label{eq:lorentz-eom}
  \dv{\vb{p}}{t} = -e \left[ \vb{E}(\vb{r}, t) + \vb{v} \cp \vb{B}(\vb{r}, t) \right]\,.
\end{equation}

\subsection{Non-relativistic treatment}

\subsection{Relativistic treatment}

\section{The Ponderomotive Force}

\subsection{Non-relativistic treatment}

\subsection{Relativistic treatment}

\end{document}
